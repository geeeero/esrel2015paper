%narms.tex, an example driver file for Balkema documents.

%use the following for A4 paper:
\documentclass[12pt,a4paper,twocolumn,fleqn]{narmsabs}

% packages needed
\usepackage{subfigure}
\usepackage{epsfig}
\usepackage{timesmt}

% add here more packages based on the document format

\usepackage[utf8]{inputenc}
\usepackage[T1]{fontenc}
\usepackage{graphicx}
\usepackage[english]{babel}

\usepackage{amsmath}
\usepackage{amsfonts}
\usepackage{amssymb}
\usepackage{amsthm}
\usepackage{bm}

\usepackage[usenames,dvipsnames]{xcolor}
\usepackage{url}

\usepackage[bookmarks]{hyperref}

% setting math equation indent from left 0pts

\mathindent=0pt%

% use this for chicaco style reference
% Author references
% IMPORTANT: Author wants to format references in chicaco style Author must use BiBTex
% IMPORTANT: Author wants to format numbered references remove chicaco style file and \bibliographystyle{chicaco}

\usepackage{chicaco}

%  \cite{key}
%    which produces citations with full author list and year.
%    eg. (Brown 1978; Jarke, Turner, Stohl, et al. 1985)

%  \citeNP{key}
%    which produces citations with full author list and year, but without
%    enclosing parentheses:
%    eg. Brown 1978; Jarke, Turner & Stohl 1985

%  \citeA{key}
%    which produces citations with only the full author list.
%    eg. (Brown; Jarke, Turner & Stohl)

%  \citeANP{key}
%    which produces citations with only the full author list, without
%    parentheses eg. Brown; Jarke, Turner & Stohl

%  \citeN{key}
%    which produces citations with the full author list and year, but
%    can be used as nouns in a sentence; no parentheses appear around
%    the author names, but only around the year.
%      eg. Shneiderman (1978) states that......
%    \citeN should only be used for a single citation.

%  \shortcite{key}
%    which produces citations with abbreviated author list and year.

%  \shortciteNP{key}
%    which produces citations with abbreviated author list and year.

%  \shortciteA{key}
%    which produces only the abbreviated author list.

%  \shortciteANP{key}
%    which produces only the abbreviated author list.

%  \shortciteN{key}
%    which produces the abbreviated author list and year, with only the
%    year in parentheses. Use with only one citation.

%  \citeyear{key}
%    which produces the year information only, within parentheses.

%  \citeyearNP{key}
%    which produces the year information only.


%%%%%%%%%%%%%%%%%%%%%%%%%%%%%%%%%%%%%%%%%%%%%%%%%%%%%%%%%%%%%%%%%%%
%%%  All this stuff is from modifying the article.cls for Balkema
%%%  specifications.

%\title{...}
%\author{...}
%use \aff for author affiliations
% use \authornext for from second author
% empty line space between multiple authors
%\abstract{...}
%\maketitle{}

%%%%%%% Style for TABLES
% insert tabular command inside \tabletext{} this will produce tables in 10pts

\begin{document}
\title{Robust Bayesian Estimation of System Reliability\\ for Scarce and Surprising Data}
\author{{Gero Walter} \\
{\aff{School of Industrial Engineering}} \\
{\aff{Eindhoven University of Technology, Eindhoven, The Netherlands}} \\
\\
{\authornext{Andrew Graham \& Frank P.A. Coolen}}\\
{\aff{Department of Mathematical Sciences}} \\
{\aff{Durham University, Durham, United Kingdom.}}}
\maketitle


\section*{Abstract}

In reliability engineering, data about failure events is often scarce.
To arrive at meaningful estimates for the reliability of a system,
it is therefore often necessary to also include expert information in the analysis,
which is straightforward in the Bayesian approach by using an informative prior distribution.

A problem that then can arise is called prior-data conflict (see, e.g., \citeNP{2006:evans}):
from the viewpoint of the prior, the observed data seem very surprising,
i.e., the information from data is in conflict with the prior assumptions.
It has been recognised that models based on conjugate priors can be insensitive to prior-data conflict,
in the sense that the spread of the posterior distribution does not increase in case of such a conflict
(see, e.g., \citeNP[\S A.1.2]{diss} for two examples),
thus conveying a false sense of certainty by communicating that we can quantify the reliability of a system quite precisely
when in fact we cannot.

As was shown by \citeN{Walter2009a}, models using sets of conjugate priors %$\MZ$
(generated through sets of canonical parameters) can mitigate this issue.
In this approach, which can be seen as a robust Bayesian procedure or imprecise probability method,
uncertainty quantifications like, e.g., the reliability at time $t$,
are expressed through intervals or ranges instead of single numbers,
such that the shorter the intervals, the more precise the corresponding probability statements.
(See \citeNP[\S 3.1 and \S 3.2]{diss} for the general framework and its comparison with other models based on sets of priors.)

We use this approach to estimate the reliability of a simplified parallel system.
We exemplarily consider the problem of forecasting the reliability of a currently running new or one of a kind system,
where we have vague prior information on the lifetimes of the components the system is made of,
and the only available data consist of observed behaviour of the system components so far,
that is, the failure times of the components that have already failed,
and the fact the remaining components still function,
whose failure time is thus right-censored.

We model vague prior knowledge on component lifetimes through a set of prior distributions,
and study how surprisingly early or late component failures
affect the prediction of the reliability of the system.
We show that we get indeed more cautious probability statements in case of prior-data conflict,
while obtaining more precise reliability bounds when prior and data are in agreement,
such that the posterior system survival function (or other posterior inferences)
adequately reflect the state of affairs.

The parameters through which prior information is encoded have a clear interpretation
and are thus easily elicited;
calculation of lower and upper predictive reliability bounds is tractable,
requiring only a simple two-dimensional box-constrained optimization.

%***They lead to wider sets of posteriors, %$\MN$,
%and thus to more cautious uncertainty quantifications, in case of prior-data conflict,
%while giving more precise quantifications when prior and data are in agreement.

\bibliographystyle{chicaco}
\bibliography{esrel2015refs}

\end{document}

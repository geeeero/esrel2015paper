\documentclass[authoryear]{elsarticle}

\usepackage[utf8]{inputenc}
\usepackage[T1]{fontenc}
\usepackage{graphicx}
\usepackage[english]{babel}

\usepackage{amsmath}
\usepackage{amsfonts}
\usepackage{amssymb}
\usepackage{amsthm}
\usepackage{bm}

\usepackage[bookmarks]{hyperref}


% -----------------------------------------------------------------------------

% definitions


% -----------------------------------------------------------------------------

\title{Robust Bayesian System Reliability Estimation for Scarce and Surprising Data}

\author[eindh]{Gero Walter}
\ead{g.m.walter@tue.nl}
\author[durham]{Andrew Graham}
\ead{andrew.graham@durham.ac.uk}
\author[durham]{Frank P.A.~Coolen}
\ead{frank.coolen@durham.ac.uk}

\address[eindh]{School of Industrial Engineering, TU Eindhoven, Eindhoven, NL}
\address[durham]{Department of Mathematical Sciences, Durham University, Durham, UK}

\begin{document}

\maketitle

\begin{abstract}
In reliability engineering, data about failure events is often scarce.
To arrive at meaningful estimates for the reliability of a system,
it is therfore often necessary to also include expert information in the analysis,
which can be easily done in the Bayesian approach by using an informative prior distribution.

A problem that then can arise is called prior-data conflict:
from the viewpoint of the prior, the observed data seem very surprising,
i.e., the information from data is in conflict with the prior assumptions.
It has been recognised that models based on conjugate priors can be insensitive to prior-data conflict,
in the sense that the spread of the posterior distribution does not increase in case of such a conflict,
thus conveying a false sense of certainty by communicating that we know the reliability of a system quite precisely when in fact we do not.

We present an approach to mitigate this issue, by considering sets of prior distributions,
and making use of the recently introduced survival signature to characterise the system under study.
Our approach can be seen as a robust Bayesian procedure or imprecise probability method
that appropriately reflects surprising data in the posterior survival function or other posterior inferences.
\end{abstract}

\begin{keyword}
System Reliability \sep Survival Signature \sep Bayesian Inference \sep Prior-Data Conflict
\end{keyword}

% -----------------------------------------------------------------------------

\section{Introduction}



\section{Survival Signature}

allows to separate system structure from individual component reliability (time aspect only in components),
several type of components


\section{Robust Bayesian Inference}

prior on lifetime distribution parameters (also test data for components?),
observed failures in system surprising,
effect on reliability/survival function.

prior-data conflict, sets of priors,
generalized iLUCK models, 

Andrew's results


\section{Conclusion and Outlook}


multiple type of components, nonparametric lifetime, censored observations 


\end{document}


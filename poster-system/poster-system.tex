\documentclass[a0,red]{opacpstr}
%
% Add additionally required packages
%
\usepackage{amssymb, amsmath, amsfonts, enumerate}
\usepackage{multicol}

\usepackage{tikz}
\usetikzlibrary{%
   arrows,%
   calc,%
   fit,%
   patterns,%
   plotmarks,%
   shapes.geometric,%
   shapes.misc,%
   shapes.symbols,%
   shapes.arrows,%
   shapes.callouts,%
   shapes.multipart,%
%   shapes.gates.logic.US,%
%   shapes.gates.logic.IEC,%
   er,%
   automata,%
   backgrounds,%
   chains,%
   topaths,%
   trees,%
   petri,%
%   mindmap,%
   matrix,%
%   calendar,%
   folding,%
   fadings,%
   through,%
   positioning,%
   scopes,%
   decorations.fractals,%
   decorations.shapes,%
   decorations.text,%
   decorations.pathmorphing,%
   decorations.pathreplacing,%
   decorations.footprints,%
   decorations.markings,%
   shadows}

%
% Add additional macros
%
\newcommand{\vcenterbox}[1]{\ensuremath{\vcenter{\hbox{#1}}}}
%
% Define title, authors and possibly logo's of partners/sponsors
% For logo's of partners/sponsors, a box of width 52mm and height 18mm is
% available. Notice that TU/e poster format has been defined for A3 (explaining
% the 52mm and 18mm).
%
\title{System Reliability Estimation\\ under Prior-Data Conflict}
\author{{\large Gero Walter${}^\text{a}$, Frank P.A. Coolen${}^\text{b}$, Simme Douwe Flapper${}^\text{a}$}\\
        {\small ${}^\text{a}$ School of Industrial Engineering, Eindhoven University of Technology, Eindhoven, NL\\
                ${}^\text{b}$ Department of Mathematical Sciences, Durham University, Durham, UK}}
\graphicspath{{logos/}}
\otherlogos{\includegraphics[width=55mm]{logounidurham-large}}%
%            \hfill%
%            \includegraphics[width=23mm]{logo_stw}%
%           }



\begin{document}

%\begin{figure*}[ht]%
\fbox{%
\includegraphics[width=\textwidth]{logounidurham-large}%
}%
%\end{figure*}

\begin{multicols}{3}
\section{Before making your poster}
Before you start to make your poster, first quickly walk through the corridor.
When you arrive at the end, think of posters that caught your attention. Then
think of why these posters caught your attention. Most likely this will be
posters with many colorful pictures and not too much text. Make sure your
poster will be noticable!

Also, try to explain to someone in ten minutes about your work using pen and
paper. What pictures did you draw? What results did you present? Those pictures
should become the figures on your poster.  Using only those pictures you should
be able to explain your work and your results. A rule of thumb is to have four
figures.

%\section{Getting started: TU/e font required}
%The mnposter style requires correct installation of the TU/e font and logo.
%This is guaranteed when installing MiKTex 2.9 which can be obtained from
%\verb=\\webmath1\MiKTeX29=. For Linux users, obtain from the MiKTex directory
%the file \texttt{localtexmf.zip} and add its contents to your local texmf
%directory.

\section{Options of the stylefile}
This stylefile has two kinds of options. The first kind of
options has to do with the papersize. Specify either \texttt{a4} (which is
used for the website) or \texttt{a1plus} (which is used for plotting). The
second kind of options is related to the color of the titleframe, which default
is \texttt{red}, but can also be \texttt{orange}, \texttt{green},
\texttt{cyan}, or \texttt{blue}.

So, in order to obtain a large size poster with tick marks, with a cyan
titleframe, start your document with
\verb=\documentclass[a4,cyan]{mnposter}=.


\section{Color usage}
Restrict the colors in your figures to the TU/e main colors:
\begin{description}
\item[PMS Process Cyan] \colorbox{tuepmsprocesscyan}{RGB 0,162,222}
\item[PMS Warm Red] \colorbox{tuepmswarmred}{RGB 247,49,49}
\item[PMS 206] \colorbox{tuepms206}{RGB 214,0,74}
\item[PMS 226] \colorbox{tuepms226}{RGB 214,0,123}
\item[PMS 253] \colorbox{tuepms253}{RGB 173,32,173}
\item[PMS 300] \colorbox{tuepms300}{RGB 0,102,204}
\item[PMS 2748] \colorbox{tuepms2748}{RGB 16,16,115}
\end{description}
and the TU/e support colors:
\begin{description}
\item[PMS 137] \colorbox{tuepms137}{RGB 255,154,0}
\item[PMS Yellow 012] \colorbox{tuepmsyellow012}{RGB 255,221,0}
\item[PMS 396] \colorbox{tuepms396}{RGB 206,223,0}
\item[PMS Green] \colorbox{tuepmsgreen}{RGB 0,172,130}
\item[PMS 3135] \colorbox{tuepms3135}{RGB 0,146,181}
\item[PMS 375] \colorbox{tuepms375}{RGB 132,210,0}
\end{description}

\end{multicols}

\iffalse
\section{Logos of partners/sponsors}
Logos of partners/sponsors can be added to the MN logo on the top of the
poster. For this the macro \verb=\otherlogos= has been provided. For an
example of the usage, see the \LaTeX\ source of this document. For logos of
partners/sponsors, a box of width 52mm and height 18mm is available (formatting
of a TU/e scientific poster is defined on A3).

Keep in mind that according to the TU/e corporate identity (TU/e huisstijl), at
most three logos of partners/sponsors can be used, where the (optical) size of
these logos should be less than that of the TU/e logo.

\section{Other remarks}
Most of the standard \LaTeX\ macros should be available. As an examples,
figures can be included in the standard way.
\begin{figure}[htb]
\begin{center}
%\vcenterbox{\includegraphics[width=18mm]{logo_oce}}
\hskip1cm
%\vcenterbox{\includegraphics[width=52mm]{logo_vanderlande}}
\end{center}
\caption{An example picture, in this case two more logos}
\label{fig:morelogos}
\end{figure}
You can refer to this figure as Figure~\ref{fig:morelogos}. In a similar way
also tables can be entered.

Furthermore, equations can be entered in the standard way:
\begin{subequations}
\label{eq:dynamics}
\begin{align}
\dot x(t) &= Ax(t)+Bu(t) & x(0)&=x_0 \\
y(t)      &= Cx(t)+Du(t)
\label{eq:output}
\end{align}
\end{subequations}
and you can refer to \eqref{eq:dynamics} or \eqref{eq:output}.

If desired, one can even add references, such as %\cite{LRN99b}.

\section{Final remarks/suggestions for improvement}
If you have any questions or remarks about this stylefile can be addressed to
Erjen Lefeber (Gemini Zuid, room~0.122). Keep in mind that stylefiles should
not be edited, so if you feel a need for modifying the stylefile either a
new version of the stylefile should be made available via the website, or
you should solve your problem differently.
\fi

%\bibliographystyle{plain}
%\bibliography{poster}

\end{document}

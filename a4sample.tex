%narms.tex, an example driver file for Balkema documents.

%use the following for A4 paper:
\documentclass[12pt,a4paper,twocolumn,fleqn]{narms}

% packages needed
\usepackage{subfigure}
\usepackage{epsfig}
\usepackage{timesmt}

% add here more packages based on the document format


% setting math equation indent from left 0pts

\mathindent=0pt%

% use this for chicaco style reference
% Author references
% IMPORTANT: Author wants to format references in chicaco style Author must use BiBTex
% IMPORTANT: Author wants to format numbered references remove chicaco style file and \bibliographystyle{chicaco}

\usepackage{chicaco}

%  \cite{key}
%    which produces citations with full author list and year.
%    eg. (Brown 1978; Jarke, Turner, Stohl, et al. 1985)

%  \citeNP{key}
%    which produces citations with full author list and year, but without
%    enclosing parentheses:
%    eg. Brown 1978; Jarke, Turner & Stohl 1985

%  \citeA{key}
%    which produces citations with only the full author list.
%    eg. (Brown; Jarke, Turner & Stohl)

%  \citeANP{key}
%    which produces citations with only the full author list, without
%    parentheses eg. Brown; Jarke, Turner & Stohl

%  \citeN{key}
%    which produces citations with the full author list and year, but
%    can be used as nouns in a sentence; no parentheses appear around
%    the author names, but only around the year.
%      eg. Shneiderman (1978) states that......
%    \citeN should only be used for a single citation.

%  \shortcite{key}
%    which produces citations with abbreviated author list and year.

%  \shortciteNP{key}
%    which produces citations with abbreviated author list and year.

%  \shortciteA{key}
%    which produces only the abbreviated author list.

%  \shortciteANP{key}
%    which produces only the abbreviated author list.

%  \shortciteN{key}
%    which produces the abbreviated author list and year, with only the
%    year in parentheses. Use with only one citation.

%  \citeyear{key}
%    which produces the year information only, within parentheses.

%  \citeyearNP{key}
%    which produces the year information only.


%%%%%%%%%%%%%%%%%%%%%%%%%%%%%%%%%%%%%%%%%%%%%%%%%%%%%%%%%%%%%%%%%%%
%%%  All this stuff is from modifying the article.cls for Balkema
%%%  specifications.

%\title{...}
%\author{...}
%use \aff for author affiliations
% use \authornext for from second author
% empty line space between multiple authors
%\abstract{...}
%\maketitle{}

%%%%%%% Style for TABLES
% insert tabular command inside \tabletext{} this will produce tables in 10pts


\begin{document}
\title{Weakly nonlinear dynamics of dunes}
\author{{M. Colombini \& A. Stocchino} \\
{\aff{Dipartimento di Ingegneria delle Costruzioni, dell'Ambiente e del Territorio}} \\
{\aff{Universit\`a di Genova, Genova, Italy}} \\
\\
{\authornext{R. Schielen}}\\
{\aff{Department of Water Engineering and Management}} \\
{\aff{University of Twente, Enschede, The Netherlands.}}}

\date{}% No date.

\abstract{Weakly nonlinear analyses have proved their validity in
the field of morphodynamic instability to describe the evolution
of finite amplitude perturbations of the bed topography. In a
recent study, Colombini and Stocchino (2008) analyzed
theoretically the case of dunes and antidunes that develop, under
suitable conditions, in an infinitely wide open channel with an
erodible bottom composed by uniform sediments. By introducing a
slow timescale in the analysis, they derived an amplitude equation
of the Landau-Stuart (LS) type, which describes the nonlinear
evolution of a linearly unstable perturbation in the neighbourhood
of its marginal conditions. The analysis of the steady solutions
of the amplitude equation shows that, for values of the ratio of
the shear velocity to the depth-averaged velocity of practical
interest, dune bifurcation is supercritical, whereas antidune
bifurcation is subcritical.  Introducing also a slow spatial
scale, an amplitude equation of the Ginzburg-Landau (GL) type is
eventually derived, which describes the temporal as well as the
spatial modulation of marginally unstable dunes. The weakly
nonlinear dynamics of a narrow spectrum of unstable waves centered
around the critical wavenumber is then analysed, whereby Landau
theory is limited to the temporal evolution of the critical mode.
Moreover, it is possible to study the stability of GL solutions
against general perturbations in contrast to the LS theory where
only the stability against perturbations with exactly the critical
wavenumber can be analyzed. Periodic solutions of the GL amplitude
equation can either be stable, which means that a periodically
modulated pattern will emerge, or unstable. In fact, the group
velocity of the unstable wavepacket depends on the wavenumber;
therefore, local convergence and divergence of the perturbations
occurs, possibly causing the periodic solution to become unstable.
This depends on the coefficients of the Ginzburg-Landau equation,
which in turn are related to the relevant flow and sediment
parameters, namely the Froude number and the ratio of grain size
over flow depth.}

\maketitle

\section{INTRODUCTION}

\subsection{Type area}

Dunes appear in the so-called lower flow regime corresponding to
small values of the Froude number and are characterized by
downstream propagation and by being almost out of phase with
respect to water-surface gravity waves. On the contrary, antidunes
occur in the upper flow regime, i.e. for values of the Froude
number close to unity, and typically propagate upstream, being
almost in phase with free surface oscillations. In both cases, the
main geometrical mean flow depth. The idea that bedform formation
in rivers can be interpreted in terms of an instability process of
the system composed by the flow and the erodible bed dates back to
the sixties, when the first seminal studies on this subject were
published \cite{ke63,re65}. This research field is still quite
active \cite{as02} and several morphodynamic patterns have been
investigated making use of techniques imported from the field of
hydrodynamic stability. Linear analyses allow for the definition
of unstable regions in the parameter space where bedforms are
expected to form \cite[among others]{en70,fr74,cf00} and to an
indication on the wavelength and celerity of the most unstable
disturbances. No information is gathered on bedform amplitude at a
linear level, however. More recently, \shortciteN{pd07} used a
parametrization of the separation streamline to avoid modelling
the flow and sediment transport in the separation zone itself and
successfully used this concept in a model to compute dynamic
roughness due to dunes \shortcite{pd09}. Extending previous linear
studies \cite{co04,cs05}, \citeN{cs08} presented a temporal weakly
non linear analysis of dunes and antidunes.  In the former case,
an equilibrium amplitude is obtained, which compares
satisfactorily against experimental observations. In the present
contribution, the work of \citeN{cs08} is extended to describe the
spatial modulations of wave packets (which is a natural extension
of the theory).

\section{FORMULATION OF THE PROBLEM}
\label{sec:exp}

Let us consider a uniform turbulent free surface flow in a
infinitely wide straight channel. The triplet composed by the
fluid density $\rho$, the mean friction velocity $u_f^*$ and depth
$D^*$ of the unperturbed uniform flow has been used for
nondimensionalization. In the following, variables with a star
superscript are to be intended as dimensional variables.

Moreover, we define a nondimensional conductance coefficient $C$
as the ratio between the unperturbed depth-averaged velocity
$\overline{U}^*$ and the mean friction velocity $u^*_f$, which can
be related to the flow depth and the sediment diameter $d^*_s$
through the Keulegan equation \shortcite{as63} for fully rough
turbulent flow: \begin{equation}
C=\frac{\overline{U}^*}{u^*_f}=\frac{1}{\kappa}\ln{\left(\frac{11.09
D^*}{2.5 d^*_s}\right)}, \label{keurough} \end{equation} where
$\kappa$ is the Von K\'arm\'an constant, taken as 0.4, and the
roughness height has been set equal to $2.5 d^*_s$ after
\shortcite{eh67}.


\begin{figure}
\centerline{
\includegraphics[width=8cm]{sketch.eps}
}
\caption{Sketch of flow configuration}
\label{f:sketch1}
\end{figure}

A sketch of the coordinate system adopted is shown in
figure~\ref{f:sketch1}, where the flow is bounded between the two
lines $y=R(x,t)$ and $y=R(x,t)+D(x,t)$, with $D$ the local flow
depth. The lower boundary is set at the reference level $R$, where the
velocity is assumed to vanish.

The differential system (\ref{reyn}) is associated with
an appropriate set of kinematic and dynamic boundary conditions at the
domain boundaries. The Reynolds stresses are modelled through a
Boussinesq closure that implies the evaluation of an algebraic eddy
viscosity ($\nu_t$), based on the mixing length approach.


The function $\Phi$ is known to depend on a dimensionless form of
the bed shear stress, namely the Shields stress $\theta_b$.
Results are only moderately affected by the choice of a particular
form for the function $\Phi$. In the following, the classical
\citeN{mp48} formula: \begin{equation} \Phi = A_m (\theta_b -
\theta_c)^{\frac{3}{2}} \qquad \theta_b \geq \theta_c, \label{mpm}
\end{equation} has been employed, where $\theta_c$ is the critical Shields
stress for incipient motion. The values of $\theta_c$ and $A_m$
have been set equal to 0.0495 and 3.97, respectively, in
accordance with the corrections proposed by \citeN{wp06} in their
revisitation of the work of \citeN{mp48}. In addition, the effect
of gravity on the grain motion is included by setting the critical
Shields stress $\theta_c$ equal to: \begin{equation} \theta_c =
0.0495 - \mu (S - R,_x), \label{effective} \end{equation} where
$\mu$ is a dimensionless constant set equal to 0.1 after
\cite{fr74}.

Finally, the transformation $(x,y,t) \rightarrow (\xi,\eta,\tau)$ is
introduced that maps the flow domain into a rectangular domain.


\section{LINEAR THEORY}
\label{ltheory}

In this section we briefly summarize the essential steps of the linear
theory, referring for a detailed description to \citeN{cs05}.  The
analysis is performed in terms of normal modes, which implies that a
generic function is expanded as:
\begin{equation}
G(\xi, \eta, \tau) = G_0(\eta) + \epsilon G_1(\xi, \eta, \tau),
\end{equation}
where $\epsilon$ is a small parameter.

\subsection{Section 2}

Moreover, we set where $A, k$ and $\omega$ are the amplitude,
wavenumber and complex celerity of the bed perturbation,
respectively and $c.c.$ stands for complex conjugate. Expanding
each variable in the governing equations, boundary conditions and
turbulent closure model and collecting terms at the same order, we
are left at $O(\epsilon)$ with the following differential problem


\subsubsection{Section 3}
Thus, $\hbox{\bf Z}_{11}$ is expressed as a linear
combination of two linearly independent solutions of the
homogeneous initial value problem
\begin{equation}
\hbox{${\pcal L}$}{L}_{11} \hbox{\bf Z}_{11}^{(1,2)} = 0,
\label{hom}
\end{equation}
which satisfy the boundary conditions at the lower boundary, plus
particular solutions of the non-homogeneous differential systems:
\begin{equation}
\hbox{${\pcal L}$}{L}_{11} \hbox{\bf Z}_{11}^{(D)} = \hbox{\bf
D}_{11}, \qquad \hbox{${\pcal L}$}{L}_{11} \hbox{\bf Z}_{11}^{(R)}
= \hbox{\bf R}_{11},\label{reyn}
\end{equation}
again satisfying the lower boundary conditions. Without loss of
generality, the constants  $c_{11}^{(1)}$ and $c_{11}^{(2)}$ are
chosen so as to represent the amplitude of the perturbed
tangential and normal stresses at the reference level, respectively.


\begin{figure*}
\centerline{\includegraphics[width=15cm]{zoom2.eps}}
\caption{Regions of instability for dunes, antidunes and roll
waves; $C=20$, $f=0.02$. The dashed lines in the close-up pictures
correspond to the lines of maximum growth rate.}
\label{f:fralfamap}
\end{figure*}

Three separate eigenvalues display unstable regions in the $(k-F)$
space (see Figure~\ref{f:fralfamap}): two of them can be readily
associated with the formation of dunes and antidunes, while the
third describes the instability of fast sediment waves that appear
at high Froude numbers (i.e.\  $F \geq 2$) associated with the
presence of roll-waves. The antidune mode is characterized by a
small negative celerity (upstream propagation) while the dune mode
propagates downstream (positive celerity). The free surface and
the bed oscillations are found to be approximately in phase for
antidunes and out of phase for dunes, with a small lag coherent
with the corresponding direction of migration.

For each value of the coefficient $C$, two critical points can be
identified in the stability plot, say $(k_{cd}, F_{cd})$ and $(k_{ca},
F_{ca})$, which are circled in the close-up pictures of
figure~\ref{f:fralfamap}. They identify the onset of instability for
each mode, since, as the Froude number equals $F_{cd}$ ($F_{ca}$), the
basic plane bed solution loses stability towards periodic perturbations
characterized by wavenumber $k_{cd}$ ($k_{ca}$), which represent the
bedform. The critical Froude number for roll-wave instability
$F_{cr}$ is found in the long wave limit $k_{cr} \rightarrow 0 $.

\section{WEAKLY NONLINEAR THEORY}
\label{wnltheory}

We intend to investigate the weakly nonlinear evolution of the
perturbations of the flow-bed system in a neighbourhood of the points
($k_{cd}, F_{cd}$) and ($k_{ca}, F_{ca}$) shown in
figure~\ref{f:fralfamap}. We then define:
\begin{equation}
F=F_c (1 + \epsilon^2 F_2), \qquad k=k_c (1 + \epsilon k_1),
\label{frweak}
\end{equation}
where the subscript $c$ indicates either of the critical points,
whereas $F_2$ and $k_1$ are dummy parameters that define the extension
of the neighbourhood in the $F$ and $k$ directions, respectively.

In order to investigate the modulation of a basic critical wave with
wavenumber $k_c$ and celerity $\omega_c$ we employ a multiscale
perturbation technique and define a slow time scale $T$ and a slow
spatial scale $X$ such that:
\begin{equation}
T = \epsilon^2 \tau, \qquad \qquad
X = \epsilon (\xi - c_g \tau)
\end{equation}
where $c_g$ is the group velocity of the wave packet. Derivatives with
respect to $\tau$ and $\xi$ become, respectively:
\begin{equation}
\frac{{\rm \partial} {}}{{\rm \partial} \tau} \rightarrow
\frac{{\rm \partial} {}}{{\rm \partial} \tau} - \epsilon c_g
\frac{{\rm \partial} {}}{{\rm \partial} T}  + \epsilon^2.
\end{equation}
\begin{equation}
\frac{{\rm \partial} {}}{{\rm \partial} \xi} \rightarrow  +
\epsilon \frac{{\rm \partial} {}}{{\rm
\partial} X}
\frac{{\rm
\partial} {}}{{\rm \partial} \xi}
\end{equation}

We then expand the solution in the form:
\begin{equation}
G(\xi, \eta, \tau, X, T) = G_0 + \epsilon G_1 + \epsilon^2 G_2 +
\epsilon^3 G_3,
\label{expansion}
\end{equation}
and collect terms at the various order of approximation in $\epsilon$.

\subsection{$O(\epsilon^1)$}

At the linear level, the structure of the solution is analogous to
(\ref{pert}):
\begin{equation}
G_1 = A(X,T) G_{11} E_1 + c.c.,\label{pert}
\end{equation}
where, in general:
\begin{equation}
E_n= \exp [ n i k_c (\xi - \omega_c \tau)],\label{diffsystem1}
\end{equation}
and the complex function $A(X,T)$ is now a slowly varying function (in time and space) to
be determined. The differential system (\ref{diffsystem1}) is
recovered, with $\omega = \omega_c$ and $k = k_c$. As expected no
information is gathered on the amplitude $A$ at this level of
approximation.

\subsection{$O(\epsilon^2)$}

The structure of the solution at second order reads:
\begin{eqnarray}
G_2 &=& \{ A^2 G_{22} E_2 + c.c.\} + |A|^2 G_{20} + F_2 G_{20F} +
\nonumber \\
&+& \{(A,_X G_{21X} +A_2 G_{11} ) E_1  + c.c.\}
\label{f2exp}
\end{eqnarray}


Finally, note that in (\ref{f2exp}) a second, unknown amplitude
function $A_2$ appears. The latter is introduced as a consequence of
the non-uniqueness of the solution as provided by the solvability
condition we used for the determination of the group velocity
$c_g$. It should be noted that $A_2$ is unimportant for the subsequent
analysis and can be left undetermined.

\subsection{$O(\epsilon^3)$}

At third order the spatial dependence of the fundamental is
reproduced and therefore we can write:
\begin{equation}
G_3 = G_{31} E_1 + c.c..
\end{equation}
and the related differential system reads:
\begin{eqnarray}
\hbox{${\pcal L}$}{L}_{11} \hbox{\bf Z}_{31} &=& D_{31} \hbox{\bf
D}_{11} + R_{31} \hbox{\bf R}_{11} +
 \\
&+& A,_T \hbox{\bf P}_{31}^{(1)} + |A|^2 A \hbox{\bf P}_{31}^{(3)}
+ A,_{XX} \hbox{\bf P}_{31}^{(4)} \nonumber \label{diffsystem31}
\end{eqnarray}
where the vectors $\hbox{\bf P}_{31}^{(1,3,4)}$ are functions of
$\eta$ expressed in terms of products of basic, leading and second
order components of the perturbations.

Once the particular solutions $\hbox{\bf Z}_{31}^{(P1,P3,P4)}$ of
the non-homogeneous differential systems:
\begin{equation}
\hbox{${\pcal L}$}{L}_{11} \hbox{\bf Z}_{31}^{(P1,P3,P4)} =
\hbox{\bf P}^{(1,3,4)}_{31},\label{boundsystem21X}
\end{equation}
are obtained, the boundary conditions at the free surface and the
Exner equation can be cast in a similar way as (\ref{boundsystem21X})
to give:
\begin{eqnarray}
\hbox{\bf U}_{11}\cdot \hbox{\bf C}_{31} &=&
 A F_2 \hbox{\bf U}_{31}^{(2)}
+ |A|^2A \hbox{\bf U}_{31}^{(3)} + \label{boundsystem31}
\\
&+& A,_T \hbox{\bf U}_{31}^{(1)} + A,_{XX} \hbox{\bf U}_{31}^{(4)}
\nonumber
\end{eqnarray}
where the first term on the right-hand side is generated by the
boundary conditions and by the Exner equation.

As before, since the homogeneous part of the system
(\ref{boundsystem31}) admits of a non-trivial solution, a solvability
condition has to be imposed. Having set:
\begin{equation}
\delta_i = \det(\hbox{\bf U}^{(i)}_{11}),
\end{equation}
where the array $\hbox{\bf U}^{(i)}_{11}$ is  obtained by
substituting the vector $\hbox{\bf U}^{(i)}_{31}$ into the last
column of $\hbox{\bf U}_{11}$, we find:
\begin{equation}
\delta_1 A,_T + \delta_2 F_2 A + \delta_3  |A|^2A +
\delta_4 A,_{XX} = 0,
\end{equation}
that, after some manipulations, takes the form of the complex
Ginzburg-Landau equation (CGLE):
\begin{equation}
A,_T = \alpha_1 F_2 A + \alpha_2 |A|^2 A +\alpha_3 A,_{XX}.
\label{cgle}
\end{equation}

\section{ANALYSIS OF GINZBURG-LANDAU EQUATION}
\label{ancgle}

Setting $\alpha_3=0$ in (\ref{cgle}), a Landau-Stuart equation is
recovered, which has been extensively studied for the case of dunes
and antidunes by \citeN{cs08}. We briefly summarize in the following
their main results. If the real part of the cubic coefficient
$\alpha_2$ is negative, the bifurcation is termed supercritical and an
equilibrium amplitude is eventually attained as $T \rightarrow
\infty$:
\begin{equation}
A_e=\sqrt{-\frac{\alpha_1^r F_2 }{\alpha_2^r}}.
\label{equilibrium}
\end{equation}
For realistic values of the parameter, namely the nondimensional
conductance coefficient $C$, dune instability is found to be
supercritical, whereas antidunes are consistently subcritical. In the
latter case, no information is gathered on the amplitude at the
present level of approximation.

We then limit our analysis of the CGLE to the case of dunes, and,
firstly, bring (\ref{cgle}) into standard form by means of a suitable
transormation. Substitution of
\begin{equation}
A=A_e \exp(i \frac{\alpha_1^i}{\alpha_1^r}T')A'(X',T')
\end{equation}
where
\begin{equation}
X'=\sqrt{\frac{\alpha_1^r F_2}{\alpha_3^r}} X \qquad\qquad
T'=\alpha_1^r F_2 T
\end{equation}
yields the rescaled CGLE (primes are dropped for convenience)
\begin{equation}
A,_T = A + (1+ib) A,_{XX} - (1+ic) |A|^2 A.
\label{cgler}
\end{equation}
with
\begin{equation}
b=\frac{\alpha_3^i}{\alpha_3^r} \qquad\qquad
c=\frac{\alpha_2^i}{\alpha_2^r}
\end{equation}
Note that $b>0$ and $c<0$ in the whole range of conductance
coefficients $C$ considered. This follows from an analysis of the
coefficients $\alpha_i$ from equation (\ref{cgle}) where dune
data from various experiments are used.

We consider periodic plane-wave solutions of the form
\begin{equation}
A=P \exp ( i \Theta), \qquad
\Theta= KX -\Omega T
\label{persol}
\end{equation}
where $P(X,T)$, $\Theta(X,T) \in {R}$. By substituting
(\ref{persol}) into (\ref{cgler}) and collecting the real and
imaginary parts, we obtain
\begin{equation}
P^2 = 1 - K^2, \qquad
\Theta = K X - (b K^2 + c P^2) T
\label{order0}
\end{equation}
This gives for every choice of $K$ the amplitude and shift in phase
(with respect to the critical wave) of the dunes. Due to the signs of
$b$ and $c$, it turns out that the nonlinear dunes propagate slower
when compared to the linear theory.



\begin{table}
\caption{Margin settings for A4 size paper and letter size paper.}
\tabletext{
\begin{tabular}{lllll}\hline
&\multicolumn{2}{l}{A4 size
paper}&\multicolumn{2}{l}{Letter size paper}\\[-6pt]
&\multicolumn{2}{l}{\hrulefill}&\multicolumn{2}{l}{\hrulefill}\\
Setting&cm&inches&cm&inches\\\hline
Top&1.2&0.47"&0.32&0.13"\\
Bottom&1.3&0.51"&0.42&0.17"\\
Left&1.15&0.45"&1.45&0.57"\\
Right&1.15&0.45"&1.45&0.57"\\
All other&0.0&0.0"&0.0&0.0"\\
Column width*&9.0&3.54"&9.0&3.54"\\
Column spacing*&0.7&0.28"&0.7&0.28"\\\hline
\end{tabular}}
\end{table}




\begin{table*} \caption{Margin settings for A4 size paper and
letter size paper.}
\begin{center}
\tabletext{
\begin{tabular*}{\textwidth}{@{\extracolsep\fill}lllll@{}}\hline
&\multicolumn{2}{l}{A4 size
paper}&\multicolumn{2}{l}{Letter size paper}\\[-6pt]
&\multicolumn{2}{l}{\hrulefill}&\multicolumn{2}{l}{\hrulefill}\\
Setting&cm&inches&cm&inches\\\hline
Top&1.2&0.47"&0.32&0.13"\\
Bottom&1.3&0.51"&0.42&0.17"\\
Left&1.15&0.45"&1.45&0.57"\\
Right&1.15&0.45"&1.45&0.57"\\
All other&0.0&0.0"&0.0&0.0"\\
Column width*&9.0&3.54"&9.0&3.54"\\
Column spacing*&0.7&0.28"&0.7&0.28"\\\hline
\end{tabular*}}
\end{center}
\end{table*}


Now, periodic solutions of the type (\ref{persol}) need not to be
stable. The stability can be studied by considering perturbations
of $P$ and $\Theta$:
\begin{equation}
A=[ P+\rho(X,T)] \exp [ i (\Theta+\theta(X,T))]
\label{rspert}
\end{equation}
Substitution of (\ref{rspert}) into (\ref{persol}) leads after
linearization to a system of partial differential equations for
$\rho$ and $\theta$. We then set:
\begin{eqnarray}
\rho(X,T)&=& \varrho \exp [ i (l X - \lambda T) ], \\
\theta(X,T)&=& \vartheta \exp [ i (l X - \lambda T) ],
\end{eqnarray}
and end up with the algebraic homogeneous system:
\begin{equation}
(M- i \lambda I)
\left(
\begin{array}{c}
\varrho \\ P \vartheta
\end{array}
\right)
= 0
\end{equation}
for some matrix M.
Stability of (\ref{persol}) is then reduced to an eigenvalue analysis of $M$.
This sets a condition on $K$ (the generalized Eckhaus criterion):
\begin{equation}
K^2 < \frac{1+bc}{3+2c^2+bc}
\end{equation}
as long as the Benjamin-Feir-Newell criterion
\begin{equation}
1+bc > 0
\label{bfn}
\end{equation}
holds (for details on this stability analysis, see \shortciteN{sd93}). For the present case, the latter criterion is never satisfied
in the whole range of parameters investigated. This means that none of
the periodic solutions of the type (\ref{persol}) is stable, the
Stokes wave $(K=0)$ being the last to become unstable.
For the special case of the Stokes wave, it can be shown that for the eigenvalues of $M$ holds:
\begin{equation}
\lambda^2 + 2 i \lambda (1 + l^2) - (1 + b^2)l^4
- 2  l^2 (1 + bc)  = 0,
\end{equation}

By imposing that the imaginary parts of both eigenvalues of $M$ for $K=0$ (i.e. the Stokes wave) are negative the
following condition on the wavenumber $l$ is found:
\begin{equation}
l\geq l_c = \sqrt{\frac{-2(1+bc)}{1+b^2}}.
\end{equation}
which implies that longer perturbations of the Stokes wave are the
most unstable \cite{sd78}.
In this case however, the Stokes wave itself is also unstable.

Considering a slowly modulated Stokes wave, a physical interpretation
of this stability criterion (\ref{bfn}) can be found in
\shortciteN{sd93}. Since $c<0$ for the present case, the dispersion
relation (\ref{order0}) implies a negative nonlinear correction of the
(positive) bedforms celerity that is maximum at the top of the
envelope, so that the bedforms at either side of it propagate
faster. Therefore the bedforms on the downstream side lengthen,
whereas the waves at the upstream side are shortened. This is
associated to a positive variation of the group velocity $c_g$ with
the wavenumber $k$ that implies an accumulation of energy at the top
of the envelop, a necessary condition for instability to occur.

A natural question to ask is then what pattern of dunes will evolve in
the case of an unstable Stokes wave? Due to the instability of the
Stokes wave, the Ginzburg-Landau equation does not allow for a pure
periodic pattern of dunes. Instead, it is more likely that a
quasi-periodic pattern will occur. Solutions of the Ginzburg Landau
then describe the {\it envelope} of the dune-evolution (because the
amplitude equation depends on a slow time and spatial variable).  To
see whether this indeed occurs, a spectral analysis of (\ref{cgler})
can be performed. In \shortciteN{doe91}, it is shown that, depending
on the values of the coefficients, chaotic solutions (for the
envelope) are possible, through a classical scenario of period
doublings.  This however, is not of particular importance for
field-studies of dunes. Also quasi-periodic solutions for the envelope
already give a very irregular sequence in the amplitude of dunes.
\section{CONCLUSIONS}
The linear analysis of dunes, leading to regions of instability as
depicted in figure \ref{f:fralfamap} can be extended to a weakly
nonlinear analysis in a straightforward way. For dunes, this leads
after tedious calculations to a nonlinear amplitude equation of the
Ginzburg Landau type. The bifurcation for dunes at $F_{cd}$ is
supercritical. Hence, starting with a flat bed, for decreasing Froude
number, dunes start to evolve, and their nonlinear evolution can
adequately be described by the solutions of the Ginzburg-Landau
equation. Simple periodic solutions turn out to be unstable,
however. This suggests that more complicated behaviour of dunes will
emerge: the moving dunes show an increase and decrease in amplitude as
they evolve in time and space. Their migration speed is also slightly
less than predicted by linear theory.

Although not shown in this paper, a similar analysis can be done for
anti-dunes. In that case, however, it turns out theat the bifurcation
is subcritical. This means that the flat bed looses stability for
$F>F_{ca}$, but the bifurcating solution is also unstable. In that
case the perturbation analysis of section \ref{wnltheory} must be
extended to fifth order, which eventually yields a quintic version of
the CGLE.  Besides antidunes, there are other phenomena in nature that
exhibit a similar behaviour (Plane Poiseuille flow and Taylor flow,
for instance). \citeN{ei89} have studied the general case of
degenerate modulation equations while \citeN{de91} have looked at
periodic and quasi-periodic solutions. To find the quintic coefficient
however, requires an even larger number of computations due to
nonlinear interactions.

The weakly nonlinear analysis of dunes presented herein follows
closely the weakly nonlinear analysis of bars developed by
\shortciteN{sd93}. This is slightly remarkable because a
depth-averaged model is used in the latter, whereas depth plays an
essential role in the dynamics of dune. However, in both cases, the
main difficulty in the analysis is the determination of the nonlinear
coefficient of CGLE. Once determined, the analysis of the
Ginzburg-Landau equation itself follows rather straightforwardly.  The
nonlinear corrections of the shape lead for dunes as well as for bars
to steeper fronts and weaker slopes at the lee sides. Furthermore,
both bedforms decelerate with respect to the linear theory.  A
spectral analysis will reveal more dynamic behaviour in the evolution
of the envelope and hence of the individual dunes.


\bibliographystyle{chicaco}
\bibliography{a4sample}

\end{document}

%narms.tex, an example driver file for Balkema documents.

%use the following for A4 paper:
\documentclass[12pt,a4paper,twocolumn,fleqn]{narms}

% packages needed
\usepackage{subfigure}
\usepackage{epsfig}
\usepackage{timesmt}

% add here more packages based on the document format


% setting math equation indent from left 0pts

\mathindent=0pt%

% use this for chicaco style reference
% Author references
% IMPORTANT: Author wants to format references in chicaco style Author must use BiBTex
% IMPORTANT: Author wants to format numbered references remove chicaco style file and \bibliographystyle{chicaco}

%\usepackage{chicaco}

%  \cite{key}
%    which produces citations with full author list and year.
%    eg. (Brown 1978; Jarke, Turner, Stohl, et al. 1985)

%  \citeNP{key}
%    which produces citations with full author list and year, but without
%    enclosing parentheses:
%    eg. Brown 1978; Jarke, Turner & Stohl 1985

%  \citeA{key}
%    which produces citations with only the full author list.
%    eg. (Brown; Jarke, Turner & Stohl)

%  \citeANP{key}
%    which produces citations with only the full author list, without
%    parentheses eg. Brown; Jarke, Turner & Stohl

%  \citeN{key}
%    which produces citations with the full author list and year, but
%    can be used as nouns in a sentence; no parentheses appear around
%    the author names, but only around the year.
%      eg. Shneiderman (1978) states that......
%    \citeN should only be used for a single citation.

%  \shortcite{key}
%    which produces citations with abbreviated author list and year.

%  \shortciteNP{key}
%    which produces citations with abbreviated author list and year.

%  \shortciteA{key}
%    which produces only the abbreviated author list.

%  \shortciteANP{key}
%    which produces only the abbreviated author list.

%  \shortciteN{key}
%    which produces the abbreviated author list and year, with only the
%    year in parentheses. Use with only one citation.

%  \citeyear{key}
%    which produces the year information only, within parentheses.

%  \citeyearNP{key}
%    which produces the year information only.


%%%%%%%%%%%%%%%%%%%%%%%%%%%%%%%%%%%%%%%%%%%%%%%%%%%%%%%%%%%%%%%%%%%
%%%  All this stuff is from modifying the article.cls for Balkema
%%%  specifications.

%\title{...}
%\author{...}
%use \aff for author affiliations
% use \authornext for from second author
% empty line space between multiple authors
%\abstract{...}
%\maketitle{}

%%%%%%% Style for TABLES
% insert tabular command inside \tabletext{} this will produce tables in 10pts

\begin{document}
\title{Preparing a two column extended abstract}
\author{{A. Unknown \& C. Unknown} \\
{\aff{A.A. Balkema Publishers, Leiden, The Netherlands}} \\\\
{\authornext{B. Unknown}}\\
{\aff{New Institute, Gouda, Netherlands}}
%{\authornext{R. Schielen}}\\
%{\aff{Department of Water Engineering and Management}} \\
%{\aff{University of Twente, Enschede, The Netherlands.}}\\
} \maketitle


\section*{Abstract}

{\bf The text in this paper is for visual purpose only. No rights
can be taken from this.}

Built in 1968, this multi-level concrete parking facility has been
subjected to years of moisture and chloride ion attack causing it
to experience corrosion-related deterioration to the point that
its structural integrity was being compromised. Over the years,
moisture and chloride contamination has caused both the embedded
mild steel and paper wrapped "Button-Head" post-tensioning to
experience corrosion-related deterioration, which in turn, has
resulted in the failure of the post-tensioning system.  After
years of monitoring and structural capacity calculations, sections
of the structure were deemed to be no longer able to safely
support the applied loading.  This paper will illustrate the
procedures implemented in the design of the new post-tensioning
system, which enabled us to rejuvenate this 40-year old structure
and extend its overall effective service life.  This paper will
also outline how the new system was installed while the parking
facility remained operational.


A detailed riparian field study to assess the im-portance of
bedrock groundwater in streamflow processes was established in the
headwaters of the Afon Hafren, mid-Wales, UK. Results from this
study identified distinct groundwater horizons close to the stream
channel.  Different flow pathways and travel times resulted in a
different chemical charac-ter of groundwaters in these different
horizons.  Groundwater discharge from these horizons into the
stream was by piston displacement in response to re-charging
rainfall higher up in the catchment.  Groundwater upwelling into
the soils indicated soil water to be sourced from both groundwater
and rain-fall.  Soil waters closest to the stream (ca. 25m) were
predominantly groundwater controlled and may be the major source
for ecologically toxic soil compo-nents such as aluminium entering
the river.

A detailed riparian field study to assess the im-portance of
bedrock groundwater in streamflow processes was established in the
headwaters of the Afon Hafren, mid-Wales, UK.  Results from this
study identified distinct groundwater horizons close to the stream
channel.  Different flow pathways and travel times resulted in a
different chemical charac-ter of groundwaters in these different
horizons.


\begin{figure}
\centerline{
\includegraphics[width=8cm]{sketch.eps}
} \caption{Sketch of flow configuration} \label{f:sketch1}
\end{figure}


A detailed riparian field study to assess the im-portance of
bedrock groundwater in streamflow processes was established in the
headwaters of the Afon Hafren, mid-Wales, UK.  Results from this
study identified distinct groundwater horizons close to the stream
channel.

\begin{table}
\caption{Margin settings for A4 size paper and letter size paper.}
\tabletext{
\begin{tabular}{lllll}\hline
&\multicolumn{2}{l}{A4 size
paper}&\multicolumn{2}{l}{Letter size paper}\\[-6pt]
&\multicolumn{2}{l}{\hrulefill}&\multicolumn{2}{l}{\hrulefill}\\
Setting&cm&inches&cm&inches\\\hline
Top&1.2&0.47"&0.32&0.13"\\
Bottom&1.3&0.51"&0.42&0.17"\\
Left&1.15&0.45"&1.45&0.57"\\
Right&1.15&0.45"&1.45&0.57"\\
All other&0.0&0.0"&0.0&0.0"\\
Column width*&9.0&3.54"&9.0&3.54"\\
Column spacing*&0.7&0.28"&0.7&0.28"\\\hline
\end{tabular}}
\end{table}


\begin{thebibliography}{9}
\bibitem{} Donald, A.P. and Gee, A.S., 1992.  Acid waters in upland Wales: causes, effects and remedies. Environmental Pollu-tion, 78, 141-148.
\bibitem{} Duff, P.M.D. and Smith, A.J., 1992. Geology of England and Wales.
The Geological Society, London.
\bibitem{} Haria, A.H. and Shand, P., 2004. Evidence for deep sub-surface flow routing in forested upland Wales: implications for contaminant transport and stream flow generation. Hy-drology and Earth System Sciences, 8(3): 334-344.
\end{thebibliography}



\end{document}

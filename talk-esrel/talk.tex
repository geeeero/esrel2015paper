\documentclass{beamer}

% opacity bugfix: see http://tug.org/pipermail/pdftex/2007-December/007480.html
\pdfpageattr {/Group << /S /Transparency /I true /CS /DeviceRGB>>}

\usepackage[utf8]{inputenc}
\usepackage[OT1]{fontenc}

\usepackage{tikz}
\usetikzlibrary{%
   arrows,%
   calc,%
   fit,%
   patterns,%
   plotmarks,%
   shapes.geometric,%
   shapes.misc,%
   shapes.symbols,%
   shapes.arrows,%
   shapes.callouts,%
   shapes.multipart,%
   shapes.gates.logic.US,%
   shapes.gates.logic.IEC,%
   er,%
   automata,%
   backgrounds,%
   chains,%
   topaths,%
   trees,%
   petri,%
   mindmap,%
   matrix,%
   calendar,%
   folding,%
   fadings,%
   through,%
   patterns,%
   positioning,%
   scopes,%
   decorations.fractals,%
   decorations.shapes,%
   decorations.text,%
   decorations.pathmorphing,%
   decorations.pathreplacing,%
   decorations.footprints,%
   decorations.markings,%
   shadows}
\usepackage{animate}
\usepackage{amssymb, amsmath, amsfonts, enumerate}
%\usepackage{bbold}
\newcommand\hmmax{0}
\usepackage{bm}
%\usepackage{dsfont}
\usepackage{pxfonts}
\usepackage{xcolor}
\usepackage{url}

%\usepackage[backend=bibtex,style=authoryear,dashed=false]{biblatex}
%\addbibresource{../cv/eigene_141121.bib}
%\addbibresource{../../../_diss/v1/bib/itip-refs.bib}
%\addbibresource{../../../_diss/v1/bib/other-refs.bib}
%\addbibresource{refs-eindhoven.bib}
%\renewcommand{\bibfont}{\normalfont\scriptsize}
%\setlength{\bibhang}{3ex}

\usepackage{hyperref}

\usetheme[official=false,department=ieis]{tue2008}
%\usefonttheme{default}


%\usetheme[secheader]{Boadilla}
%\setbeamercovered{transparent}
%\setbeamercovered{invisible}
%\setbeamertemplate{navigation symbols}{}
%\setbeamertemplate{bibliography item}[text] % numbered references
%\useoutertheme{infolines}
%\setbeamertemplate{headline}{}
%\setbeamertemplate{footline}{\hspace*{5mm}\hfill\insertframenumber\hspace*{5mm}\vspace{3mm}}
%\setbeamercolor{alerted text}{fg=orange!80!black}

\definecolor{unidurham}{RGB}{126,49,123}

%\def\then{{\color{green}$\rule[0.35ex]{2ex}{0.5ex}\!\!\!\blacktriangleright$}}
\def\then{{\structure{$\rule[0.35ex]{2ex}{0.5ex}\!\!\!\blacktriangleright$}}}

\def\play{{\structure{$\blacktriangleright$}}}

%\def\blau#1{{\color{lmugreen2}#1}}
\def\rot#1{{\color{red}#1}}
\def\gruen#1{{\color{blue}#1}}

% ------------------------------------------------------------------------------

\newcommand{\vcenterbox}[1]{\ensuremath{\vcenter{\hbox{#1}}}}
%
\newcommand{\reals}{\mathbb{R}}
\newcommand{\posreals}{\reals_{>0}}
\newcommand{\posrealszero}{\reals_{\ge 0}}
\newcommand{\naturals}{\mathbb{N}}

\newcommand{\dd}{\,\mathrm{d}}

\newcommand{\mbf}[1]{\mathbf{#1}}
\newcommand{\bs}[1]{\boldsymbol{#1}}
\renewcommand{\vec}[1]{{\bm#1}}

\newcommand{\uz}{^{(0)}} % upper zero
\newcommand{\un}{^{(n)}} % upper n
\newcommand{\ui}{^{(i)}} % upper i
\newcommand{\uell}{^{(\ell)}} % upper ell

\newcommand{\ul}[1]{\underline{#1}}
\newcommand{\ol}[1]{\overline{#1}}

\newcommand{\Rsys}{R_\text{sys}}
\newcommand{\lRsys}{\ul{R}_\text{sys}}
\newcommand{\uRsys}{\ol{R}_\text{sys}}

\newcommand{\Fsys}{F_\text{sys}}
\newcommand{\lFsys}{\ul{F}_\text{sys}}
\newcommand{\uFsys}{\ol{F}_\text{sys}}

\def\Tsys{T_\text{sys}}

\newcommand{\E}{\operatorname{E}}
\newcommand{\V}{\operatorname{Var}}
\newcommand{\sd}{\operatorname{sd}}
\newcommand{\wei}{\operatorname{Wei}} % Weibull Distribution
\newcommand{\ig}{\operatorname{IG}}   % Inverse Gamma Distribution

\def\yz{y\uz}
\def\yn{y\un}
%\def\yi{y\ui}
\def\yell{y\uell}
\newcommand{\yfun}[1]{y^{({#1})}}
\newcommand{\yfunl}[1]{\ul{y}^{({#1})}}
\newcommand{\yfunu}[1]{\ol{y}^{({#1})}}

\def\ykz{y\uz_k}
\def\ykn{y\un_k}

\def\yzl{\ul{y}\uz}
\def\yzu{\ol{y}\uz}
\def\ynl{\ul{y}\un}
\def\ynu{\ol{y}\un}
\def\yil{\ul{y}\ui}
\def\yiu{\ol{y}\ui}

\def\ykzl{\ul{y}\uz_k}
\def\ykzu{\ol{y}\uz_k}
\def\yknl{\ul{y}\un_k}
\def\yknu{\ol{y}\un_k}


\def\nz{n\uz}
\def\nn{n\un}
%\def\ni{n\ui}
\def\nell{n\uell}
\newcommand{\nfun}[1]{n^{({#1})}}
\newcommand{\nfunl}[1]{\ul{n}^{({#1})}}
\newcommand{\nfunu}[1]{\ol{n}^{({#1})}}

\def\nkz{n\uz_k}
\def\nkn{n\un_k}
\newcommand{\nkzfun}[1]{n\uz_{#1}}


\def\nzl{\ul{n}\uz}
\def\nzu{\ol{n}\uz}
\def\nnl{\ul{n}\un}
\def\nnu{\ol{n}\un}
\def\nil{\ul{n}\ui}
\def\niu{\ol{n}\ui}

\def\nkzl{\ul{n}\uz_k}
\def\nkzu{\ol{n}\uz_k}
\def\nknl{\ul{n}\un_k}
\def\nknu{\ol{n}\un_k}


\def\taut{\tau(\vec{t})}
\def\ttau{\tilde{\tau}}
\def\ttaut{\ttau(\vec{t})}

\def\PZ{\mathrm I\!\Pi\uz}
\def\PN{\mathrm I\!\Pi\un}

\def\MZ{\mathcal{M}\uz}
\def\MN{\mathcal{M}\un}

\def\MkZ{\mathcal{M}\uz_k}
\def\MkN{\mathcal{M}\un_k}

\def\PkZ{\Pi\uz_k}
\def\PkN{\Pi\un_k}
\newcommand{\PZi}[1]{\Pi\uz_{#1}}


\def\tnow{t_\text{now}}
\def\tpnow{t^+_\text{now}}

\newcommand{\comp}[1]{\raisebox{-2mm}{\tikz{\node[%type1
rectangle,rounded corners=0mm,draw,fill=tuepmsgreen!70,thick,inner sep=0pt,minimum size=6mm]{#1};}}}

\newcommand{\cyansec}[1]{\textcolor{tuecyan}{\large\bf #1}}
\newcommand{\cyanalert}[1]{\textcolor{tuecyan}{#1}}

% ------------------------------------------------------------------------------


\title{Robust Bayesian Estimation of System Reliability with Scarce and Surprising Data}

\author{Gero Walter\inst{1}, Andrew Graham\inst{2}, Frank Coolen\inst{2}}
\institute{ \inst{1} Eindhoven University of Technology, Eindhoven, NL\\ 
            \inst{2} Durham University, Durham, UK \\[2ex]
            \url{g.m.walter@tue.nl} \\[2ex]
            \includegraphics[height=9mm]{logos/tuelogo} \quad
            \includegraphics[height=9mm]{logos/logounidurham-large} \quad
            \includegraphics[height=9mm]{logos/dinalog-hp} }
\date{09-09-2015}

\begin{document}

\frame{
\titlepage
}

\iffalse
\frame{\frametitle{Outline}
\begin{enumerate}
\item System reliability with the survival signature
\item Cautious modelling of prior information
\end{enumerate}
}
\fi

%\section{System Reliability}


\frame{
\frametitle{Setting: a one of a kind parallel system}

\begin{columns}
\begin{column}{0.2\textwidth}
\begin{tikzpicture}
[type1/.style={rectangle,draw,fill=tuepmsgreen!70,very thick,inner sep=0pt,minimum size=6mm},
 type2/.style={rectangle,draw,fill=tuepmsgreen!70,very thick,inner sep=0pt,minimum size=6mm},
 type3/.style={rectangle,draw,fill=tuepmsgreen!70,very thick,inner sep=0pt,minimum size=6mm},
 cross/.style={cross out,draw=red,very thick,minimum width=7mm, minimum height=5mm},
 hv path/.style={thick, to path={-| (\tikztotarget)}},
 vh path/.style={thick, to path={|- (\tikztotarget)}}]
\coordinate (start) at (0,0);
\coordinate (prllst) at (0.3,0);
\node[type1] (p1) at ( 1, 1.5) {};
\node[type1] (p2) at ( 1, 0.5) {};
%\node[type1] (p3) at ( 1,-0.5) {};
\node[inner sep=0pt,minimum size=6mm] (p3) at ( 1,-0.5) {$\vdots$};
\node[type1] (p4) at ( 1,-1.5) {};
\node at ($(p1.south east) + (0.1,-0.0)$) {$1$};
\node at ($(p2.south east) + (0.1,-0.0)$) {$2$};
\node at ($(p4.south east) + (0.1,-0.0)$) {$\ell$};
\coordinate (prllen) at (1.7,0);
\coordinate (end)   at (2,0);
\path (start)  edge          (prllst)
      (prllst) edge[vh path] (p1.west)
               edge[vh path] (p2.west)
               edge[vh path] (p3.west)
               edge[vh path] (p4.west)
      (prllen) edge[vh path] (p1.east)
               edge[vh path] (p2.east)
               edge[vh path] (p3.east)
               edge[vh path] (p4.east)
               edge          (end);
\end{tikzpicture}\\
($1$ out of $\ell$)
\end{column}
\begin{column}{0.8\textwidth}
We want to learn about the system reliability $\Rsys(t) = P(\Tsys > t)$ based on
\begin{enumerate}
\item[\play] system run until time $\tnow$:
 \begin{itemize}
 \item[] $\ell$ observations, each being either\\ a failure time $t_j$ or a censoring time $t_j^+ = \tnow$
 \end{itemize}
\item[\play] cautious assumptions on component reliability:
 \begin{itemize}
 \item[] expert information,\\ e.g. from the component manufactor\\ \alert{which we don't trust entirely}
 \end{itemize}
\end{enumerate}
How to combine these two information sources?
\end{column}
\end{columns}
}

\frame{
\frametitle{Bayesian inference}
\begin{align*}
\begin{array}{ccccl}
\text{expert info}        & + & \text{data}       & \to & \text{complete picture} \\[1.5ex]
\text{prior distribution} & + & \text{likelihood} & \to & \text{posterior distribution} \\[1.5ex]
p(\lambda) & \times & p_c(\mbf{t} \mid \lambda) & \propto & p(\lambda \mid \mbf{t}) \qquad\text{\cyanalert{\play\ Bayes' Rule}} \\
\downarrow & & \downarrow & & \hspace*{3ex} \downarrow \\
\text{inverse Gamma} & & \text{Weibull with} & & \text{inverse Gamma} \\
\text{prior}         & & \text{fixed shape $k$}  & & \quad \text{posterior} \quad \text{\cyanalert{\play\ conjugacy}} \\[1ex]
\lambda \sim \ig(\alpha\uz,\beta\uz) & & \mbf{t}\mid\lambda \sim \wei_k(\lambda) & & \lambda\mid\mbf{t} \sim \ig(\alpha\uell,\beta\uell)
%{\textstyle \lambda \sim \ig(\nz\!+\!1,\nz\yz)}
\end{array}
\end{align*}
\begin{itemize}
\item conjugacy holds also for censored observations
\item makes learning about component reliability tractable,\\  %posterior distribution
      just update parameters:\quad $\alpha\uz \to \alpha\uell$, $\beta\uz \to \beta\uell$
\item closed form for system reliability function $\Rsys(t\mid\mbf{t})$
\end{itemize}
}

\frame{
\frametitle{Prior-data conflict}
What if expert information and data tell different stories?\\
\play\ reparametrization helps to understand effect of prior-data conflict:\\
\begin{tikzpicture}
[pfeil/.style={-latex', line width=1mm, color=tuered, shorten <=1mm},
 cyanrand/.style={rounded corners, text centered, draw=tuecyan!50, inner sep=1mm, thick},
 redbrace/.style={draw=tuered, decoration=brace, decorate, line width=0.8mm},
 redbox/.style={text centered, draw=tuered, inner sep=1mm, very thick}]
\node at (0,0) {\parbox[c]{\textwidth}{%
\begin{align*}
\nz &= \alpha\uz - 1\,,
&
\yz &= \beta\uz / (\alpha\uz - 1)
%\end{align*}
\intertext{%
such that
$\lambda \sim \ig(\nz+1,\nz\yz)$ and $\lambda\mid\mbf{t} \sim \ig(\nell+1,\nell\yell)$,
where}%
%\begin{align*}
\nell &= \nz + \ell\,, 
&
\yell &=  \frac{\nz}{\nz + \ell} \, \yz + \frac{\ell}{\nz + \ell} \cdot \frac{1}{\ell} \textstyle\sum_{j=1}^\ell t_j^k
\end{align*}
}};
\node[cyanrand] (yz) at (-0.5,-2.6) {$\yz = \E[\lambda]$};
\draw [pfeil] (yz.north east) to [out= 30,in=260] (1.1,-1.35);
\draw [pfeil] (yz.north west) to [out=130,in=240] (-1.3,0.9);
\node[cyanrand] (nz) at (-4  ,-2.6) {$\nz =$ pseudocounts};
\draw [pfeil] (nz.north)      to [out=80,in=250] (-3.7,-1.3);
\draw [pfeil] (nz.north west) to [out=95,in=240] (-4.7, 0.9);
\node[cyanrand] (ml) at ( 4.3,-2.6) {ML estimator $\hat{\lambda}$};
\draw [redbrace] (4.9,-1.8) to (3.4,-1.8);
\node[redbox] (wavg) at (0,-3.5) {$\E[\lambda\mid\mbf{t}]$ is a weighted average of $\E[\lambda]$ and $\hat{\lambda}$!};
\end{tikzpicture}
}

\frame{
\frametitle{Prior-data conflict example}
\begin{tikzpicture}
[pfeil/.style={-latex', line width=1mm, color=tuered, shorten <=1mm},
 cyanrand/.style={rounded corners, text centered, draw=tuecyan!50, inner sep=1mm, thick},
 redbrace/.style={draw=tuered, decoration=brace, decorate, line width=0.8mm},
 redbox/.style={text centered, draw=tuered, inner sep=1mm, very thick}]
\uncover<1-3>{\node at (0,0) {\includegraphics[width=0.95\textwidth]{pdc1}};}
\uncover<1-3>{\node at (2,1) {\color{tuered} $\yz = \E[\lambda] = 103$ (9 weeks), $\sd = 146$};}
\uncover<2-3>{\node at (0,0) {\includegraphics[width=0.95\textwidth]{pdc2}};}
\uncover<2-3>{\node at (-3.6,0) {$t_1 = 1$}; \node at (-3.6,-0.5) {$t_2 = 2$}; \node at (-3.6,-1) {$\hat{\lambda} = 2.5$}; }
\uncover<3-3>{\node at (0,0) {\includegraphics[width=0.95\textwidth]{pdc3}};}
\uncover<3-3>{\node at (2,2) {\color{tuecyan} $\yfun{2} = \E[\lambda\mid\mbf{t}] = 53$ (6.4 weeks), $\sd = 61$};}
%
\uncover<4->{\node at (0,0) {\includegraphics[width=0.95\textwidth]{nopdc1}};}
\uncover<4->{\node at (2,1) {\color{tuered} $\yz = \E[\lambda] = 62$ (7 weeks), $\sd = 72$};}
\uncover<5->{\node at (0,0) {\includegraphics[width=0.95\textwidth]{nopdc2}};}
\uncover<5->{\node at (-2.1,0) {$t_1 = 6$}; \node at (-2.1,-0.5) {$t_2 = 7$}; \node at (-2.1,-1) {$\hat{\lambda} = 42.5$}; }
\uncover<6->{\node at (0,0) {\includegraphics[width=0.95\textwidth]{nopdc3}};}
\uncover<6->{\node at (2,2) {\color{tuecyan} $\yfun{2} = \E[\lambda\mid\mbf{t}] = 52$ (6.4 weeks), $\sd = 61$};}
\uncover<6->{\node at (2,1.6) {\color{tuecyan} \emph{almost the same as before!}};}
\end{tikzpicture}
}

\frame{
\frametitle{Imprecise / interval probability}
\begin{itemize}
\item Averaging property holds \emph{for all conjugate models} (!)\\
\cyanalert{Can we mitigate this and still keep tractability?}
\item Reliability function $R(t)$ is a collection of probability statements:\\
%In the long run, what percentage of components survive past $t$?\\
The probability that the system survives past $t$.\\
\cyanalert{How can we express uncertainty\\ about these probability statements?}
\item[\play] \textbf{Add \alert{imprecision} as new modelling dimension:\\
\alert{Sets of priors} model uncertainty in probability statements\\
and allow to better model partial or vague information on $\lambda$.}
\item This can also be seen as systematic sensitivity analysis
or robust Bayesian approach.
\end{itemize}
}

\frame{
\frametitle{Sets of prior distributions}
\uncover<1->{%
\begin{block}{Uncertainty about probability statements}
\centerline{smaller sets $=$ more precise probability statements}
\vspace*{1ex}
\parbox[t]{0.45\textwidth}{\centering \textbf{Lottery A}\\
                          Number of winning tickets:\\
                          exactly known as 5 out of 100\\
                          \then\ $P(\text{win}) = 5/100$}
\qquad
\parbox[t]{0.45\textwidth}{\centering \textbf{Lottery B}\\
                          Number of winning tickets:\\
                          not exactly known, supposedly\\
                          between 1 and 7 out of 100\\
                          \then\ $P(\text{win}) = [1/100,\, 7/100]$}
\end{block}}
\uncover<2->{%
Let parameters $(\nz, \yz)$ vary in a set $\PZ$ \then\ set of priors\\[1ex]
Walter \& Augustin (2009), Walter (2013): $\PZ = [\nzl, \nzu] \times [\yzl, \yzu]$\\
allows meaningful reaction to prior-data conflict:
\begin{itemize}
\item larger set of posteriors %$\to$ larger intervals
\item more imprecise / cautious probability statements
\end{itemize}
sets of priors $\to$ sets of posteriors by updating element by element\\
GBR (Walley 1991) ensures \emph{coherence} {\small (a consistency property)}
}
}

\frame{
\frametitle{Sets of prior distributions: example}
\begin{tikzpicture}
[pfeil/.style={-latex', line width=1mm, color=tuered, shorten <=1mm},
 cyanrand/.style={rounded corners, text centered, draw=tuecyan!50, inner sep=1mm, thick},
 redbrace/.style={draw=tuered, decoration=brace, decorate, line width=0.8mm},
 redbox/.style={text centered, draw=tuered, inner sep=1mm, very thick}]
\uncover<1-1>{\node at (0,0) {\includegraphics[width=0.95\textwidth]{lucknopdc1}};}
\uncover<2-2>{\node at (0,0) {\includegraphics[width=0.95\textwidth]{lucknopdc2}};}
\uncover<3-3>{\node at (0,0) {\includegraphics[width=0.95\textwidth]{luckpdc1}};}
\uncover<4-4>{\node at (0,0) {\includegraphics[width=0.95\textwidth]{luckpdc2}};}
\uncover<5-5>{\node at (0,0) {\includegraphics[width=0.95\textwidth]{pdc-nopdc}};}
\end{tikzpicture}
}

\frame{
\frametitle{System reliability}

\play\ Closed form for the parallel system reliability:
\begin{multline*}
\Rsys(t \mid \vec{t}_m^\ell, \nz, \yz) \\
  = 1 - \sum_{i=0}^{\ell-m} (-1)^i {\ell-m \choose i} \, 
   \left( \frac{\nz \yz + (\ell-m)   t_\text{now}^k + \sum_{j=1}^{m} t_j^k}%
               {\nz \yz + (\ell-m-i) t_\text{now}^k + \sum_{j=1}^{m} t_j^k + i t^k} \right)^{\nz + m + 1}
\end{multline*}
\play\ Lower / upper bound through optimization for each $t$:
\begin{align*}
\lRsys(t \mid \vec{t}_m^\ell, \PZ) &= \min_{\nz \in [\nzl, \nzu]} \Rsys(t \mid \vec{t}_m^\ell, \nz, \yzl) \\ 
\uRsys(t \mid \vec{t}_m^\ell, \PZ) &= \max_{\nz \in [\nzl, \nzu]} \Rsys(t \mid \vec{t}_m^\ell, \nz, \yzu)
\end{align*}
}

\frame{
\frametitle{system reliability graphs}

}

\frame{
\frametitle{summary, extensions}

}

\frame{
\frametitle{literature}

}

\end{document}





\documentclass{beamer}

% opacity bugfix: see http://tug.org/pipermail/pdftex/2007-December/007480.html
\pdfpageattr {/Group << /S /Transparency /I true /CS /DeviceRGB>>}

\usepackage[utf8]{inputenc}
\usepackage[OT1]{fontenc}

\usepackage{tikz}
\usetikzlibrary{%
   arrows,%
   calc,%
   fit,%
   patterns,%
   plotmarks,%
   shapes.geometric,%
   shapes.misc,%
   shapes.symbols,%
   shapes.arrows,%
   shapes.callouts,%
   shapes.multipart,%
   shapes.gates.logic.US,%
   shapes.gates.logic.IEC,%
   er,%
   automata,%
   backgrounds,%
   chains,%
   topaths,%
   trees,%
   petri,%
   mindmap,%
   matrix,%
   calendar,%
   folding,%
   fadings,%
   through,%
   patterns,%
   positioning,%
   scopes,%
   decorations.fractals,%
   decorations.shapes,%
   decorations.text,%
   decorations.pathmorphing,%
   decorations.pathreplacing,%
   decorations.footprints,%
   decorations.markings,%
   shadows}
\usepackage{animate}
\usepackage{amssymb, amsmath, amsfonts, enumerate}
%\usepackage{bbold}
\newcommand\hmmax{0}
\usepackage{bm}
%\usepackage{dsfont}
\usepackage{pxfonts}
\usepackage{xcolor}
\usepackage{url}

%\usepackage[backend=bibtex,style=authoryear,dashed=false]{biblatex}
%\addbibresource{../cv/eigene_141121.bib}
%\addbibresource{../../../_diss/v1/bib/itip-refs.bib}
%\addbibresource{../../../_diss/v1/bib/other-refs.bib}
%\addbibresource{refs-eindhoven.bib}
%\renewcommand{\bibfont}{\normalfont\scriptsize}
%\setlength{\bibhang}{3ex}

\usepackage{hyperref}

\usetheme[official=false,department=ieis]{tue2008}
%\usefonttheme{default}


%\usetheme[secheader]{Boadilla}
%\setbeamercovered{transparent}
%\setbeamercovered{invisible}
%\setbeamertemplate{navigation symbols}{}
%\setbeamertemplate{bibliography item}[text] % numbered references
%\useoutertheme{infolines}
%\setbeamertemplate{headline}{}
%\setbeamertemplate{footline}{\hspace*{5mm}\hfill\insertframenumber\hspace*{5mm}\vspace{3mm}}
%\setbeamercolor{alerted text}{fg=orange!80!black}

\definecolor{unidurham}{RGB}{126,49,123}

%\def\then{{\color{green}$\rule[0.35ex]{2ex}{0.5ex}\!\!\!\blacktriangleright$}}
\def\then{{\structure{$\rule[0.35ex]{2ex}{0.5ex}\!\!\!\blacktriangleright$}}}

\def\play{{\structure{$\blacktriangleright$}}}

%\def\blau#1{{\color{lmugreen2}#1}}
\def\rot#1{{\color{red}#1}}
\def\gruen#1{{\color{blue}#1}}

% ------------------------------------------------------------------------------

\newcommand{\vcenterbox}[1]{\ensuremath{\vcenter{\hbox{#1}}}}
%
\newcommand{\reals}{\mathbb{R}}
\newcommand{\posreals}{\reals_{>0}}
\newcommand{\posrealszero}{\reals_{\ge 0}}
\newcommand{\naturals}{\mathbb{N}}

\newcommand{\dd}{\,\mathrm{d}}

\newcommand{\mbf}[1]{\mathbf{#1}}
\newcommand{\bs}[1]{\boldsymbol{#1}}
\renewcommand{\vec}[1]{{\bm#1}}

\newcommand{\uz}{^{(0)}} % upper zero
\newcommand{\un}{^{(n)}} % upper n
\newcommand{\ui}{^{(i)}} % upper i

\newcommand{\ul}[1]{\underline{#1}}
\newcommand{\ol}[1]{\overline{#1}}

\newcommand{\Rsys}{R_\text{sys}}
\newcommand{\lRsys}{\ul{R}_\text{sys}}
\newcommand{\uRsys}{\ol{R}_\text{sys}}

\newcommand{\Fsys}{F_\text{sys}}
\newcommand{\lFsys}{\ul{F}_\text{sys}}
\newcommand{\uFsys}{\ol{F}_\text{sys}}

\def\Tsys{T_\text{sys}}

\newcommand{\E}{\operatorname{E}}
\newcommand{\V}{\operatorname{Var}}
\newcommand{\wei}{\operatorname{Wei}} % Weibull Distribution
\newcommand{\ig}{\operatorname{IG}}   % Inverse Gamma Distribution

\def\yz{y\uz}
\def\yn{y\un}
%\def\yi{y\ui}
\newcommand{\yfun}[1]{y^{({#1})}}
\newcommand{\yfunl}[1]{\ul{y}^{({#1})}}
\newcommand{\yfunu}[1]{\ol{y}^{({#1})}}

\def\ykz{y\uz_k}
\def\ykn{y\un_k}

\def\yzl{\ul{y}\uz}
\def\yzu{\ol{y}\uz}
\def\ynl{\ul{y}\un}
\def\ynu{\ol{y}\un}
\def\yil{\ul{y}\ui}
\def\yiu{\ol{y}\ui}

\def\ykzl{\ul{y}\uz_k}
\def\ykzu{\ol{y}\uz_k}
\def\yknl{\ul{y}\un_k}
\def\yknu{\ol{y}\un_k}


\def\nz{n\uz}
\def\nn{n\un}
%\def\ni{n\ui}
\newcommand{\nfun}[1]{n^{({#1})}}
\newcommand{\nfunl}[1]{\ul{n}^{({#1})}}
\newcommand{\nfunu}[1]{\ol{n}^{({#1})}}

\def\nkz{n\uz_k}
\def\nkn{n\un_k}
\newcommand{\nkzfun}[1]{n\uz_{#1}}


\def\nzl{\ul{n}\uz}
\def\nzu{\ol{n}\uz}
\def\nnl{\ul{n}\un}
\def\nnu{\ol{n}\un}
\def\nil{\ul{n}\ui}
\def\niu{\ol{n}\ui}

\def\nkzl{\ul{n}\uz_k}
\def\nkzu{\ol{n}\uz_k}
\def\nknl{\ul{n}\un_k}
\def\nknu{\ol{n}\un_k}


\def\taut{\tau(\vec{t})}
\def\ttau{\tilde{\tau}}
\def\ttaut{\ttau(\vec{t})}

\def\MZ{\mathcal{M}\uz}
\def\MN{\mathcal{M}\un}

\def\MkZ{\mathcal{M}\uz_k}
\def\MkN{\mathcal{M}\un_k}

\def\PkZ{\Pi\uz_k}
\def\PkN{\Pi\un_k}
\newcommand{\PZi}[1]{\Pi\uz_{#1}}


\def\tnow{t_\text{now}}
\def\tpnow{t^+_\text{now}}

\newcommand{\comp}[1]{\raisebox{-2mm}{\tikz{\node[%type1
rectangle,rounded corners=0mm,draw,fill=tuepmsgreen!70,thick,inner sep=0pt,minimum size=6mm]{#1};}}}

\newcommand{\cyansec}[1]{\textcolor{tuecyan}{\large\bf #1}}

% ------------------------------------------------------------------------------


\title{Robust Bayesian Estimation of System Reliability with Scarce and Surprising Data}

\author{Gero Walter\inst{1}, Andrew Graham\inst{2}, Frank Coolen\inst{2}}
\institute{ \inst{1} Eindhoven University of Technology, Eindhoven, NL\\ 
            \inst{2} Durham University, Durham, UK \\[2ex]
            \url{g.m.walter@tue.nl} \\[2ex]
            \includegraphics[height=9mm]{logos/tuelogo} \quad
            \includegraphics[height=9mm]{logos/logounidurham-large} \quad
            \includegraphics[height=9mm]{logos/dinalog-hp} }
\date{09-09-2015}

\begin{document}

\frame{
\titlepage
}

\iffalse
\frame{\frametitle{Outline}
\begin{enumerate}
\item System reliability with the survival signature
\item Cautious modelling of prior information
\end{enumerate}
}
\fi

%\section{System Reliability}


\frame{
\frametitle{Setting: a one of kind parallel system}

\begin{columns}
\begin{column}{0.2\textwidth}
\begin{tikzpicture}
[type1/.style={rectangle,draw,fill=tuepmsgreen!70,very thick,inner sep=0pt,minimum size=6mm},
 type2/.style={rectangle,draw,fill=tuepmsgreen!70,very thick,inner sep=0pt,minimum size=6mm},
 type3/.style={rectangle,draw,fill=tuepmsgreen!70,very thick,inner sep=0pt,minimum size=6mm},
 cross/.style={cross out,draw=red,very thick,minimum width=7mm, minimum height=5mm},
 hv path/.style={thick, to path={-| (\tikztotarget)}},
 vh path/.style={thick, to path={|- (\tikztotarget)}}]
\coordinate (start) at (0,0);
\coordinate (prllst) at (0.3,0);
\node[type1] (p1) at ( 1, 1.5) {};
\node[type1] (p2) at ( 1, 0.5) {};
%\node[type1] (p3) at ( 1,-0.5) {};
\node[inner sep=0pt,minimum size=6mm] (p3) at ( 1,-0.5) {$\vdots$};
\node[type1] (p4) at ( 1,-1.5) {};
\node at ($(p1.south east) + (0.1,-0.0)$) {$1$};
\node at ($(p2.south east) + (0.1,-0.0)$) {$2$};
\node at ($(p4.south east) + (0.1,-0.0)$) {$\ell$};
\coordinate (prllen) at (1.7,0);
\coordinate (end)   at (2,0);
\path (start)  edge          (prllst)
      (prllst) edge[vh path] (p1.west)
               edge[vh path] (p2.west)
               edge[vh path] (p3.west)
               edge[vh path] (p4.west)
      (prllen) edge[vh path] (p1.east)
               edge[vh path] (p2.east)
               edge[vh path] (p3.east)
               edge[vh path] (p4.east)
               edge          (end);
\end{tikzpicture}\\
($1$ out of $\ell$)
\end{column}
\begin{column}{0.8\textwidth}
We want to learn about the system reliability $\Rsys(t) = P(\Tsys > t)$ based on
\begin{enumerate}
\item[\play] system run until time $\tnow$:
 \begin{itemize}
 \item[] $\ell$ observations, each being either\\ a failure time $t_j$ or a censoring time $t_j^+ = \tnow$
 \end{itemize}
\item[\play] cautious assumptions on component reliability:
 \begin{itemize}
 \item[] expert information,\\ e.g. from the component manufactor\\ \alert{which we don't trust entirely}
 \end{itemize}
\end{enumerate}
How to combine these two information sources?
\end{column}
\end{columns}
}

\frame{
\frametitle{Bayesian inference}
\begin{align*}
\begin{array}{ccccl}
\text{expert info}        & + & \text{data}       & \to & \text{complete picture} \\
\text{prior distribution} & + & \text{likelihood} & \to & \text{posterior distribution} \\
p(\lambda) & \times & p_c(\mbf{t} \mid \lambda) & \propto & p(\lambda \mid \mbf{t}) \qquad\text{(Bayes' Rule)} \\
\downarrow & & \downarrow & & \hspace*{3ex} \downarrow \\
\text{inverse Gamma} & & \text{Weibull with} & & \text{inverse Gamma} \\
\text{prior}         & & \text{fixed shape}  & & \quad \text{posterior} \quad \text{(conjugacy)} \\
{\textstyle \lambda \sim \ig(\nz\!+\!1,\nz\yz)}
\end{array}
\end{align*}
}

\end{document}





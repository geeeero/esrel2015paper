%narms.tex, an example driver file for Balkema documents.

%use the following for A4 paper:
\documentclass[12pt,a4paper,twocolumn,fleqn]{narms}

% packages needed
\usepackage{subfigure}
\usepackage{epsfig}
\usepackage{timesmt}

% add here more packages based on the document format

\usepackage[utf8]{inputenc}
\usepackage[T1]{fontenc}
\usepackage{graphicx}
\usepackage[english]{babel}

\usepackage{amsmath}
\usepackage{amsfonts}
\usepackage{amssymb}
\usepackage{amsthm}
\usepackage{bm}

\usepackage[usenames,dvipsnames]{xcolor}
\usepackage{url}

\usepackage[bookmarks]{hyperref}

% setting math equation indent from left 0pts

\mathindent=0pt%

% use this for chicaco style reference
% Author references
% IMPORTANT: Author wants to format references in chicaco style Author must use BiBTex
% IMPORTANT: Author wants to format numbered references remove chicaco style file and \bibliographystyle{chicaco}

\usepackage{chicaco}

%  \cite{key}
%    which produces citations with full author list and year.
%    eg. (Brown 1978; Jarke, Turner, Stohl, et al. 1985)

%  \citeNP{key}
%    which produces citations with full author list and year, but without
%    enclosing parentheses:
%    eg. Brown 1978; Jarke, Turner & Stohl 1985

%  \citeA{key}
%    which produces citations with only the full author list.
%    eg. (Brown; Jarke, Turner & Stohl)

%  \citeANP{key}
%    which produces citations with only the full author list, without
%    parentheses eg. Brown; Jarke, Turner & Stohl

%  \citeN{key}
%    which produces citations with the full author list and year, but
%    can be used as nouns in a sentence; no parentheses appear around
%    the author names, but only around the year.
%      eg. Shneiderman (1978) states that......
%    \citeN should only be used for a single citation.

%  \shortcite{key}
%    which produces citations with abbreviated author list and year.

%  \shortciteNP{key}
%    which produces citations with abbreviated author list and year.

%  \shortciteA{key}
%    which produces only the abbreviated author list.

%  \shortciteANP{key}
%    which produces only the abbreviated author list.

%  \shortciteN{key}
%    which produces the abbreviated author list and year, with only the
%    year in parentheses. Use with only one citation.

%  \citeyear{key}
%    which produces the year information only, within parentheses.

%  \citeyearNP{key}
%    which produces the year information only.

%%%%%%%%%%%%%%%%%%%%%%%%%%%%%%%%%%%%%%%%%%%%%%%%%%%%%%%%%%%%%%%%%%%
%%%  All this stuff is from modifying the article.cls for Balkema
%%%  specifications.

%\title{...}
%\author{...}
%use \aff for author affiliations
% use \authornext for from second author
% empty line space between multiple authors
%\abstract{...}
%\maketitle{}

%%%%%%% Style for TABLES
% insert tabular command inside \tabletext{} this will produce tables in 10pts


\newcommand{\reals}{\mathbb{R}}
\newcommand{\posreals}{\reals_{>0}}
\newcommand{\posrealszero}{\reals_{\ge 0}}
\newcommand{\naturals}{\mathbb{N}}

\newcommand{\dd}{\,\mathrm{d}}

\newcommand{\mbf}[1]{\mathbf{#1}}
\newcommand{\bs}[1]{\boldsymbol{#1}}
\renewcommand{\vec}[1]{{\bm#1}}

\newcommand{\uz}{^{(0)}} % upper zero
\newcommand{\un}{^{(n)}} % upper n
\newcommand{\ui}{^{(i)}} % upper i

\newcommand{\ul}[1]{\underline{#1}}
\newcommand{\ol}[1]{\overline{#1}}

\newcommand{\lRsys}{\ul{R}_\text{sys}}
\newcommand{\uRsys}{\ol{R}_\text{sys}}

\newcommand{\lFsys}{\ul{F}_\text{sys}}
\newcommand{\uFsys}{\ol{F}_\text{sys}}

\newcommand{\E}{\operatorname{E}}
\newcommand{\V}{\operatorname{Var}}
\newcommand{\ig}{\operatorname{IG}}   % Inverse Gamma Distribution

\def\yz{y\uz}
\def\yn{y\un}
%\def\yi{y\ui}
\newcommand{\yfun}[1]{y^{({#1})}}

\def\yzl{\ul{y}\uz}
\def\yzu{\ol{y}\uz}
\def\ynl{\ul{y}\un}
\def\ynu{\ol{y}\un}
\def\yil{\ul{y}\ui}
\def\yiu{\ol{y}\ui}

\def\nz{n\uz}
\def\nn{n\un}
%\def\ni{n\ui}
\newcommand{\nfun}[1]{n^{({#1})}}

\def\nzl{\ul{n}\uz}
\def\nzu{\ol{n}\uz}
\def\nnl{\ul{n}\un}
\def\nnu{\ol{n}\un}
\def\nil{\ul{n}\ui}
\def\niu{\ol{n}\ui}

\def\taut{\tau(\vec{t})}
\def\ttau{\tilde{\tau}}
\def\ttaut{\ttau(\vec{t})}


\newcommand{\comments}[1]{{\small\color{gray} #1}}


\begin{document}
\title{Robust Bayesian System Reliability Estimation\\ for Scarce and Surprising Data}

\author{{Gero Walter} \\
{\aff{School of Industrial Engineering}} \\
{\aff{TU Eindhoven, Eindhoven, The Netherlands}} \\
\\
{\authornext{Frank P.A. Coolen \& Andrew Graham}}\\
{\aff{Department of Mathematical Sciences}} \\
{\aff{Durham University, Durham, United Kingdom.}}}

\date{}% No date.

\abstract{In reliability engineering, data about failure events is often scarce.
To arrive at meaningful estimates for the reliability of a system,
it is therfore often necessary to also include expert information in the analysis,
which can be easily done in the Bayesian approach by using an informative prior distribution.

A problem that then can arise is called prior-data conflict:
from the viewpoint of the prior, the observed data seem very surprising,
i.e., the information from data is in conflict with the prior assumptions.
It has been recognised that models based on conjugate priors can be insensitive to prior-data conflict,
in the sense that the spread of the posterior distribution does not increase in case of such a conflict,
thus conveying a false sense of certainty by communicating that we know the reliability of a system quite precisely when in fact we do not.

We present an approach to mitigate this issue, by considering sets of prior distributions,
and making use of the recently introduced survival signature to characterise the system under study.
Our approach can be seen as a robust Bayesian procedure or imprecise probability method
that appropriately reflects surprising data in the posterior survival function or other posterior inferences.
}


\maketitle

\section{INTRODUCTION}

In reliability engineering, a central task is to describe the reliability of a complex system.
This is usually done by determining the \emph{reliability function} $R(t)$,
in other contexts also known as the \emph{survival function} $S(t)$,
giving the probability that the system has not failed by time $t$:
\begin{align}
R_\text{sys}(t) = S_\text{sys}(t) = P(T_\text{sys} \geq t)\,,
\end{align}
where $T_\text{sys}$ is the random variable giving the failure time of the system. %
%\footnote{}
Based on the distribution of $T_\text{sys}$, which can also be expressed
in terms of the cumulative distribution function $F_\text{sys}(t) = 1 - R_\text{sys}(t)$,
the density $p_\text{sys}(t)$ or the hazard rate $\lambda_\text{sys}(t)$,
decisions about, e.g., scheduling of maintenance work can be made.

Often, there is no failure data for the system itself (e.g., if the system is a prototype),
but some data about failure times exists for the components the system is made of.
Information on the distribution $p_i(t)$ of component failure times $T_i$,
where $i = 1, \ldots, l$ if the system consists of $l$ components,
can then be used to derive the distribution of $T_\text{sys}$.
In this paper, we will mostly consider a simple parallel system
where all components are of the same type,
and our observations consist solely of the failure times of the components in this system.
In the last section, we will briefly describe
how the approach can be generalized to arbitrary system structures
using the survival signature \cite{2012:survsign},
which allows for different types of components and
the inclusion of data from component tests, as in \citeNP{2014:bayessurvsign}.

We assume a parametric probability distribution for component lifetimes $T_i$. 
The Bayesian approach allows to base estimation of the component failure distributions
on both data and further knowledge not given by the data,
the latter usually in form of expert knowledge.
This knowledge is encoded in form of a so-called prior distribution,
a distribution on the parameter(s) of the component lifetime distribution.
This expert knowledge is especially important when there is very few data on the components (like in our scenario),
as only with its help meaningful estimates for the system reliability can be made.

However, the specific choice for the prior distribution to encode the given expert knowledge is often debatable,
and a specific choice of prior is difficult to justify.
A way to deal with this is to employ sensitivity analysis,
i.e., studying the effect of different choices of prior on the quantities of interest
(in our case, the reliability function, which, in Bayesian terms, is a predictive distribution).
This idea has been explored in systematic sensitivity analysis, or robust Bayesian methods
(for an overview on this approach, see, e.g.,
\shortciteNP{1994:berger}, \citeNP{2005:ruggeri}, \citeNP{2000:rios}, or \citeNP{2000:bergerinsuaruggeri}).
\comments{less citations? specific citation for robust Bayes in reliability???}

The work we present here can be seen as belonging to the robust Bayesian approach,
as our work uses sets of priors. However, our focus and interpretation is slightly different,
as we consider the result of our procedure, sets of reliability functions, as the proper result,
while a robust Bayesian would base his analyses on a single reliability function from the set
in case (s)he was able to conclude that quantities of interest are not `too sensitive' to the choice of prior.
Instead, our viewpoint is rooted in the theory of imprecise or interval probability \cite{1991:walley,2011:IESS-ip,itip},
where sets of distributions are used to express the precision of probability statements themselves:
the smaller the set, the more precise the probability statement.
Indeed, the system reliability function $R_\text{sys}(t)$ is a collection of probability statements,
and a small set for $R_\text{sys}(t)$ will indicate that we know the reliability of a system quite precisely,
while a large set will indicate that our knowledge about $T\text{sys}$ is rather shaky.
\comments{Here or at end of intro? leave out?}

In line with imprecise or interval probability methods, we will thus have, for each $t$,
a lower reliability $\lRsys(t) = \ul{P}(T_\text{sys} \geq t)$,
and an upper reliability $\uRsys(t) = \ol{P}(T_\text{sys} \geq t)$.
We will explain in Section~\ref{sec:modforsurpr} how these bounds are obtained
based on a set of prior distributions on the parameter of the component lifetime distribution.

The central merit of our method is that it adequately reflects prior-data conflict
(see, e.g., \citeNP{2006:evans}),
i.e.\ the conflict that can arise between prior assumptions on component lifetimes
and observed behaviour of components in the system under study.
As we will show in Section~\ref{sec:weibull}, when taking the standard choice of a conjugate prior,
prior-data conflict is ignored, as the spread of the posterior distribution does not increase in case of such a conflict,
ultimately conveying a false sense of certainty
by communicating that we know the reliability of a system quite precisely when in fact we do not.

In contrast, our method will indicate prior-data conflict by wider reliability bounds.
This behaviour is obtained by a specific choice for the set of priors \cite{Walter2009a}

***systematic approach: IP, see, e.g, \citeNP{1991:pericchi}

***predictive distribution!

This paper is organized as follows: *** 


\section{BAYESIAN ANALYSIS OF WEIBULL LIFETIMES}
\label{sec:weibull}

***first precise, then pdc illustration, then IP.\\

For each of the component lifetimes $T_i$, $i=1,\ldots,l$,
we assume a Weibull distribution with fixed shape parameter $k > 0$, with density and cdf%
\footnote{Our approach would be possible also for other parametric lifetime distributions
that form a canonical exponential family
(see, e.g., \citeNP[p.~202 and 272f]{2000:bernardosmith}, or \citeNP[p.~8]{diss}).}
\begin{align}
p_i(t \mid \lambda) &= \frac{k}{\lambda} t^{k-1} e^{-\frac{t^k}{\lambda}}\,, 
&
F_i(t \mid \lambda) &= 1 - e^{-\frac{t^k}{\lambda}} \,,
\end{align}
where $\lambda > 0$ and $t > 0$.%
\footnote{The shape parameter $k$ determines whether the hazard rate is increasing ($k > 1$)
or decreasing ($k < 1$) over time.
For $k=1$, we obtain the Exponential distribution with constant hazard rate as a special case.
The value for $k$ will thus often be clear from practical considerations.}
The scale parameter $\lambda$ can be interpreted through the relation
\begin{align}
\E[T_i \mid \lambda] &= \lambda^{1/k}\, \Gamma(1 + 1/k)\,.
\label{eq:lambdainterpret}
\end{align}
For encoding expert knowledge about the reliability of the components,
we need to assign a prior distribution over the scale parameter $\lambda$.
A convenient choice is to use the inverse Gamma distribution,
commonly parametrized in terms of the hyperparameters $\alpha$ and $\beta$:
\begin{align}
p(\lambda \mid \alpha, \beta) &= \frac{\beta^\alpha}{\Gamma(\alpha)} \lambda^{-(\alpha + 1)} e^{-\beta/\lambda} \,,
\end{align}
in short, $\lambda \mid \alpha, \beta \sim \ig(\alpha, \beta)$.
The inverse Gamma is convenient because it is a conjugate prior,
i.e., the posterior obtained by Bayes' rule is again inverse Gamma and thus easily tractable.

In the standard Bayesian approach, 
one has to fix a prior by choosing values for $\alpha$ and $\beta$
to encode specific prior information about component lifetimes.
In our imprecise approach, we allow instead these parameters
to vary in a set, this is advantageous also
because expert knowledge is often vague,
and it is difficult for the expert(s) to pin down precise hyperparameter values.
For the definition of the hyperparameter set,
we use however a parametrization in terms of $\nz$ and $\yz$ instead of $\alpha$ and $\beta$,
\begin{align}
\nz &= \alpha - 1\,,
&
\yz &= \beta / \nz \,,
\end{align}
where $\yz$ can be interpreted as the prior guess for the scale parameter $\lambda$,
as $\E[\lambda\mid\nz,\yz] = \yz$.
This parametrization also makes the nature of the combination
of prior information and data through Bayes' rule more clear:
After observing $n$ component lifetimes $\vec{t} = (t_1, \ldots, t_n)$,
the updated parameters are
\begin{align}
\nn &= \nz + n\,, 
&
\yz &=  \frac{\nz \yz + \taut}{\nz + n}\,,
\label{eq:ig-update}
\end{align}
where $\taut = \sum_{i=1}^n t_i^k$.%
\footnote{We thus have
$\lambda \mid \nz, \yz, \vec{t} \sim \ig(\nz + n + 1, \nz \yz + \taut)$.}
From the simple update rule \eqref{eq:ig-update}, we see that
$\yn$ is a weighted average of the prior parameter $\yz$ and the data-based quantity $\taut/n$,
with weights $\nz$ and $n$, respectively.
$\nz$ can thus be interpreted as a prior strength or pseudocount,
indicating how much our prior guess should weigh against the $n$ observations.

***$\V[\lambda\mid\nz,\yz] = $

However, the weighted average structure for $\yn$
is behind the problematic behaviour in case of prior-data conflict.
Assume that from expert knowledge we expect
to have a mean component lifetime of 9 weeks.
Using \eqref{eq:lambdainterpret}, with $k=2$ we obtain $\yz = 103.13$.
We choose $\nz = 2$, so our prior guess for the mean component lifetime
counts like having two observations with this mean.
If we now have a sample of two observations
with surprisingly early failure times $t_1 = 1$ and $t_2 = 2$,
using \eqref{eq:ig-update} we get $\nfun{2} = 4$
and $\yfun{2} = \frac{1}{4}(2 \cdot 103.13 + 1^2 + 2^2) = 52.82$,
so our posterior expectation for the scale parameter $\lambda$ is $52.82$,
equivalent to a mean component lifetime of $6.44$ weeks.
The posterior variance for $\lambda$ is ***.
Compared to the prior variance of ***,
the posterior expresses now more confidence that mean lifetimes are around $\yfun{2} = 52.82$
than the prior had about $\yz = 103.13$.
This absurd induction*** is illustrated in Figure~\ref{fig:weibull-pdc};
the posterior is shifted halfway towards the values for $\lambda$
that the two observations suggest,
and is more pointed than the prior,
thus conveying a false sense of certainty about $\lambda$.%
\footnote{This is a general problem in Bayesian analysis with canonical conjugate priors.
For such priors, the same update formula \eqref{eq:ig-update} applies,
and so conflict is averaged out, for details see \citeNP{Walter2009a} and \citeNP{diss}, \S 3.1.}

\begin{figure}

\caption{Prior and posterior for $\lambda$ given surprising observations;
the conflict between prior assumptions and data is averaged out,
with a more pointed posterior giving a false sense of certainty.}
\label{fig:weibull-pdc}
\end{figure}




\section{MODELS FOR SURPRISING DATA}
\label{sec:modforsurpr}

***still use tractability of conjugate priors, but mitigate absurd behaviour

introduce sets of priors, generalized iLUCK models, i.e.,
let $s_0$ and $t_0$ vary in rectangular set.

from set of priors set of posteriors via GBR,
will lead to set of predictive distributions (derivation in Sec~\ref{sec:andrewsresults})



\section{ROBUST BAYESIAN INFERENCE FOR A PARALLEL SYSTEM}
\label{sec:andrewsresults}

prior on lifetime distribution parameters (could include test data for components),
observed failures in system surprising,
effect on reliability/survival function.

treatment of censored observations


(selection of) Andrew's results


\section{SURVIVAL SIGNATURE (?)}

allows to separate system structure from individual component reliability (time aspect only in components),
several type of components

\section{CONCLUSION AND OUTLOOK}


multiple type of components, nonparametric lifetime as in \citeNP{2014:bayessurvsign}, other parameter set shapes 


\bibliographystyle{chicaco}
\bibliography{esrel2015refs}

\end{document}


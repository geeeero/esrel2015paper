%narms.tex, an example driver file for Balkema documents.

%use the following for A4 paper:
\documentclass[12pt,a4paper,twocolumn,fleqn]{narms}

% packages needed
\usepackage{subfigure}
\usepackage{epsfig}
\usepackage{timesmt}

% add here more packages based on the document format

\usepackage[utf8]{inputenc}
\usepackage[T1]{fontenc}
\usepackage{graphicx}
\usepackage[english]{babel}

\usepackage{amsmath}
\usepackage{amsfonts}
\usepackage{amssymb}
\usepackage{amsthm}
\usepackage{bm}

\usepackage[bookmarks]{hyperref}

% setting math equation indent from left 0pts

\mathindent=0pt%

% use this for chicaco style reference
% Author references
% IMPORTANT: Author wants to format references in chicaco style Author must use BiBTex
% IMPORTANT: Author wants to format numbered references remove chicaco style file and \bibliographystyle{chicaco}

\usepackage{chicaco}

%  \cite{key}
%    which produces citations with full author list and year.
%    eg. (Brown 1978; Jarke, Turner, Stohl, et al. 1985)

%  \citeNP{key}
%    which produces citations with full author list and year, but without
%    enclosing parentheses:
%    eg. Brown 1978; Jarke, Turner & Stohl 1985

%  \citeA{key}
%    which produces citations with only the full author list.
%    eg. (Brown; Jarke, Turner & Stohl)

%  \citeANP{key}
%    which produces citations with only the full author list, without
%    parentheses eg. Brown; Jarke, Turner & Stohl

%  \citeN{key}
%    which produces citations with the full author list and year, but
%    can be used as nouns in a sentence; no parentheses appear around
%    the author names, but only around the year.
%      eg. Shneiderman (1978) states that......
%    \citeN should only be used for a single citation.

%  \shortcite{key}
%    which produces citations with abbreviated author list and year.

%  \shortciteNP{key}
%    which produces citations with abbreviated author list and year.

%  \shortciteA{key}
%    which produces only the abbreviated author list.

%  \shortciteANP{key}
%    which produces only the abbreviated author list.

%  \shortciteN{key}
%    which produces the abbreviated author list and year, with only the
%    year in parentheses. Use with only one citation.

%  \citeyear{key}
%    which produces the year information only, within parentheses.

%  \citeyearNP{key}
%    which produces the year information only.

%%%%%%%%%%%%%%%%%%%%%%%%%%%%%%%%%%%%%%%%%%%%%%%%%%%%%%%%%%%%%%%%%%%
%%%  All this stuff is from modifying the article.cls for Balkema
%%%  specifications.

%\title{...}
%\author{...}
%use \aff for author affiliations
% use \authornext for from second author
% empty line space between multiple authors
%\abstract{...}
%\maketitle{}

%%%%%%% Style for TABLES
% insert tabular command inside \tabletext{} this will produce tables in 10pts

\begin{document}
\title{Robust Bayesian System Reliability Estimation\\ for Scarce and Surprising Data}

\author{{Gero Walter} \\
{\aff{School of Industrial Engineering}} \\
{\aff{TU Eindhoven, Eindhoven, The Netherlands}} \\
\\
{\authornext{Frank P.A. Coolen \& Andrew Graham}}\\
{\aff{Department of Mathematical Sciences}} \\
{\aff{Durham University, Durham, United Kingdom.}}}

\date{}% No date.

\abstract{In reliability engineering, data about failure events is often scarce.
To arrive at meaningful estimates for the reliability of a system,
it is therfore often necessary to also include expert information in the analysis,
which can be easily done in the Bayesian approach by using an informative prior distribution.

A problem that then can arise is called prior-data conflict:
from the viewpoint of the prior, the observed data seem very surprising,
i.e., the information from data is in conflict with the prior assumptions.
It has been recognised that models based on conjugate priors can be insensitive to prior-data conflict,
in the sense that the spread of the posterior distribution does not increase in case of such a conflict,
thus conveying a false sense of certainty by communicating that we know the reliability of a system quite precisely when in fact we do not.

We present an approach to mitigate this issue, by considering sets of prior distributions,
and making use of the recently introduced survival signature to characterise the system under study.
Our approach can be seen as a robust Bayesian procedure or imprecise probability method
that appropriately reflects surprising data in the posterior survival function or other posterior inferences.
}


\maketitle

\section{INTRODUCTION}

Bayesian analysis when not enough data, add expert knowledge,
specific choice of prior difficult to justify,
sensitivity analysis, systematic: IP,


\section{MODELS FOR SURPRISING DATA}

Model for data, here lifetimes: Weibull with fixed shape,
so need to learn about scale parameter $\lambda$.

Prior on $\lambda$: convenient choice inverse Gamma,
to fix prior, need to choose hyperparameters $\alpha$ and $\beta$.
We use different parametrization $s_0$ and $t_0$, interpretation.

Conjugacy ensures that posterior is inverse Gamma again
with updated parameters $s_n$ and $t_n$,
with simple update rule.

show problem with prior-data conflict

introduce sets of priors, generalized iLUCK models, i.e.,
let $s_0$ and $t_0$ vary in rectangular set.

Illustrate with a plot or so.

\section{ROBUST BAYESIAN INFERENCE FOR A PARALLEL SYSTEM}

prior on lifetime distribution parameters (also test data for components?),
observed failures in system surprising,
effect on reliability/survival function.

treatment of censored observations


Andrew's results


\section{SURVIVAL SIGNATURE (?)}

allows to separate system structure from individual component reliability (time aspect only in components),
several type of components

\section{CONCLUSION AND OUTLOOK}


multiple type of components, nonparametric lifetime, other parameter set shapes 


%\bibliographystyle{chicaco}
%\bibliography{a4sample}

\end{document}


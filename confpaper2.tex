%narms.tex, an example driver file for Balkema documents.

%use the following for A4 paper:
\documentclass[12pt,a4paper,twocolumn,fleqn]{narms}

% packages needed
\usepackage{subfigure}
\usepackage{epsfig}
\usepackage{timesmt}

% add here more packages based on the document format

\usepackage[utf8]{inputenc}
\usepackage[T1]{fontenc}
\usepackage{graphicx}
\usepackage[english]{babel}

\usepackage{amsmath}
\usepackage{amsfonts}
\usepackage{amssymb}
\usepackage{amsthm}
\usepackage{bm}

\usepackage[usenames,dvipsnames]{xcolor}
\usepackage{url}

\usepackage[bookmarks]{hyperref}

% setting math equation indent from left 0pts

\mathindent=0pt%

% use this for chicaco style reference
% Author references
% IMPORTANT: Author wants to format references in chicaco style Author must use BiBTex
% IMPORTANT: Author wants to format numbered references remove chicaco style file and \bibliographystyle{chicaco}

\usepackage{chicaco}

%  \cite{key}
%    which produces citations with full author list and year.
%    eg. (Brown 1978; Jarke, Turner, Stohl, et al. 1985)

%  \citeNP{key}
%    which produces citations with full author list and year, but without
%    enclosing parentheses:
%    eg. Brown 1978; Jarke, Turner & Stohl 1985

%  \citeA{key}
%    which produces citations with only the full author list.
%    eg. (Brown; Jarke, Turner & Stohl)

%  \citeANP{key}
%    which produces citations with only the full author list, without
%    parentheses eg. Brown; Jarke, Turner & Stohl

%  \citeN{key}
%    which produces citations with the full author list and year, but
%    can be used as nouns in a sentence; no parentheses appear around
%    the author names, but only around the year.
%      eg. Shneiderman (1978) states that......
%    \citeN should only be used for a single citation.

%  \shortcite{key}
%    which produces citations with abbreviated author list and year.

%  \shortciteNP{key}
%    which produces citations with abbreviated author list and year.

%  \shortciteA{key}
%    which produces only the abbreviated author list.

%  \shortciteANP{key}
%    which produces only the abbreviated author list.

%  \shortciteN{key}
%    which produces the abbreviated author list and year, with only the
%    year in parentheses. Use with only one citation.

%  \citeyear{key}
%    which produces the year information only, within parentheses.

%  \citeyearNP{key}
%    which produces the year information only.

%%%%%%%%%%%%%%%%%%%%%%%%%%%%%%%%%%%%%%%%%%%%%%%%%%%%%%%%%%%%%%%%%%%
%%%  All this stuff is from modifying the article.cls for Balkema
%%%  specifications.

%\title{...}
%\author{...}
%use \aff for author affiliations
% use \authornext for from second author
% empty line space between multiple authors
%\abstract{...}
%\maketitle{}

%%%%%%% Style for TABLES
% insert tabular command inside \tabletext{} this will produce tables in 10pts


\newcommand{\reals}{\mathbb{R}}
\newcommand{\posreals}{\reals_{>0}}
\newcommand{\posrealszero}{\reals_{\ge 0}}
\newcommand{\naturals}{\mathbb{N}}

\newcommand{\dd}{\,\mathrm{d}}

\newcommand{\mbf}[1]{\mathbf{#1}}
\newcommand{\bs}[1]{\boldsymbol{#1}}
\renewcommand{\vec}[1]{{\bm#1}}

\newcommand{\uz}{^{(0)}} % upper zero
\newcommand{\un}{^{(n)}} % upper n
\newcommand{\ui}{^{(i)}} % upper i

\newcommand{\ul}[1]{\underline{#1}}
\newcommand{\ol}[1]{\overline{#1}}

\newcommand{\Rsys}{R_\text{sys}}
\newcommand{\lRsys}{\ul{R}_\text{sys}}
\newcommand{\uRsys}{\ol{R}_\text{sys}}

\newcommand{\Fsys}{F_\text{sys}}
\newcommand{\lFsys}{\ul{F}_\text{sys}}
\newcommand{\uFsys}{\ol{F}_\text{sys}}

\newcommand{\E}{\operatorname{E}}
\newcommand{\V}{\operatorname{Var}}
\newcommand{\ig}{\operatorname{IG}}   % Inverse Gamma Distribution

\def\yz{y\uz}
\def\yn{y\un}
%\def\yi{y\ui}
\newcommand{\yfun}[1]{y^{({#1})}}
\newcommand{\yfunl}[1]{\ul{y}^{({#1})}}
\newcommand{\yfunu}[1]{\ol{y}^{({#1})}}

\def\yzl{\ul{y}\uz}
\def\yzu{\ol{y}\uz}
\def\ynl{\ul{y}\un}
\def\ynu{\ol{y}\un}
\def\yil{\ul{y}\ui}
\def\yiu{\ol{y}\ui}

\def\nz{n\uz}
\def\nn{n\un}
%\def\ni{n\ui}
\newcommand{\nfun}[1]{n^{({#1})}}
\newcommand{\nfunl}[1]{\ul{n}^{({#1})}}
\newcommand{\nfunu}[1]{\ol{n}^{({#1})}}

\def\nzl{\ul{n}\uz}
\def\nzu{\ol{n}\uz}
\def\nnl{\ul{n}\un}
\def\nnu{\ol{n}\un}
\def\nil{\ul{n}\ui}
\def\niu{\ol{n}\ui}

\def\taut{\tau(\vec{t})}
\def\ttau{\tilde{\tau}}
\def\ttaut{\ttau(\vec{t})}

\def\MZ{\mathcal{M}\uz}
\def\MN{\mathcal{M}\un}

\newcommand{\comments}[1]{{\small\color{gray} #1}}

\newtheorem{example}{Example}

\allowdisplaybreaks

\begin{document}
\title{Robust Bayesian Estimation of System Reliability\\ for Scarce and Surprising Data}

\author{{Gero Walter} \\
{\aff{School of Industrial Engineering}} \\
{\aff{Eindhoven University of Technology, Eindhoven, The Netherlands}} \\
\\
{\authornext{Andrew Graham \& Frank P.A. Coolen}}\\
{\aff{Department of Mathematical Sciences}} \\
{\aff{Durham University, Durham, United Kingdom.}}}

\date{}% No date.

\abstract{In reliability engineering, data about failure events is often scarce.
To arrive at meaningful estimates for the reliability of a system,
it is therefore often necessary to also include expert information in the analysis,
which is straightforward in the Bayesian approach by using an informative prior distribution.
%
A problem that then can arise is called prior-data conflict:
from the viewpoint of the prior, the observed data seem very surprising,
i.e., the information from data is in conflict with the prior assumptions.
It has been recognised that models based on conjugate priors can be insensitive to prior-data conflict,
in the sense that the spread of the posterior distribution does not increase in case of such a conflict,
thus conveying a false sense of certainty by communicating that we can quantify the reliability of a system quite precisely when in fact we cannot.
%
We present an approach to mitigate this issue, by considering sets of prior distributions
to model vague knowledge on component lifetimes,
and study how surprisingly early or late component failures
affect the prediction of the reliability of a simplified parallel system.
%and making use of the recently introduced survival signature to characterise the system under study.
Our approach can be seen as a robust Bayesian procedure or imprecise probability method
that appropriately reflects surprising data in the posterior system survival function or other posterior inferences.
}


\maketitle

\section{INTRODUCTION}

In reliability engineering, a central task is to describe the reliability of a complex system.
This is usually done by determining the \emph{reliability function} $R(t)$,
in other contexts also known as the \emph{survival function} $S(t)$,
giving the probability that the system has not failed by time $t$:
\begin{align}
R_\text{sys}(t) = P(T_\text{sys} \geq t)\,,
\end{align}
where $T_\text{sys}$ is the random variable giving the failure time of the system. %
%\footnote{}
Based on the distribution of $T_\text{sys}$, which can also be expressed
in terms of the cumulative distribution function $F_\text{sys}(t) = 1 - R_\text{sys}(t)$,
the density $p_\text{sys}(t)$ or the hazard rate $\lambda_\text{sys}(t)$,
decisions about, e.g., scheduling of maintenance work can be made.

Often, there is no failure data for the system itself
(e.g., if the system is a prototype, or the system is used under unique circumstances),
but some data about failure times exist for the components the system is made of.
Information on the distribution of component failure times $T_i$,
where $i = 1, \ldots, \ell$ if the system consists of $\ell$ components,
can then be used to derive the distribution of $T_\text{sys}$.
In this paper, we consider a simple parallel system
where all components are of the same type,
and our observations consist solely of the failure times of one or more components in this system.
%In the last section, we will briefly describe
%how the approach can be generalized to arbitrary system structures
%using the survival signature \cite{2012:survsign},
%which allows for different types of components and
%the inclusion of data from component tests, as in \citeNP{2014:bayessurvsign}.
%
We assume a parametric probability distribution for component lifetimes $T_i$. 
The Bayesian approach allows to base estimation of the component failure distributions
on both data and further knowledge not given by the data,
the latter usually provided in the form of expert knowledge.
This knowledge is encoded in form of a so-called prior distribution,
a distribution on the parameter(s) of the component lifetime distribution.
This expert knowledge is especially important when there is very few data on the components (like in our scenario),
as only with its help meaningful estimates for the system reliability can be made.

However, the choice for the prior distribution to encode the given expert knowledge is often debatable,
and a specific choice of prior is difficult to justify.
A way to deal with this is to employ sensitivity analysis,
i.e., studying the effect of different choices of prior distribution on the quantities of interest
(in our case, the system reliability function, which, in Bayesian terms, is a predictive distribution).
This idea has been explored in systematic sensitivity analysis, or robust Bayesian methods
(for an overview on this approach, see, e.g.,
\shortciteNP{1994:berger} or \citeNP{2000:rios}). %\citeNP{2005:ruggeri}, \citeNP{2000:bergerinsuaruggeri}

The work we present here can be seen as belonging to the robust Bayesian approach,
as our work uses sets of priors. However, our focus and interpretation is slightly different,
as we consider the result of our procedure, sets of reliability functions, as the proper result,
while a robust Bayesian would base his analyses on a single reliability function from the set
in case (s)he was able to conclude that quantities of interest are not `too sensitive' to the choice of prior.
In contrast, our viewpoint is rooted in the theory of imprecise or interval probability \cite{1991:walley,itip},
where sets of distributions are used to express the precision of probability statements themselves:
the smaller the set, the more precise the probability statement.
Indeed, the system reliability function $R_\text{sys}(t)$ is a collection of probability statements,
and a small set for $R_\text{sys}(t)$ will indicate that we can quantify the reliability of the system quite precisely,
while a large set will indicate that our knowledge about $T_\text{sys}$ is rather shaky.
%\comments{Frank: 'reliability at time $t$' where?}

In line with imprecise or interval probability methods, we will thus have, for each $t$,
a lower reliability $\lRsys(t) = \ul{P}(T_\text{sys} \geq t)$,
and an upper reliability $\uRsys(t) = \ol{P}(T_\text{sys} \geq t)$.
We will explain in Sections~\ref{sec:modforsurpr} and \ref{sec:andrewsresults} how these bounds are obtained
based on a set of prior distributions on the parameter of the component lifetime distribution.

The central merit of our method is that it adequately reflects prior-data conflict
(see, e.g., \citeNP{2006:evans}),
i.e.\ the conflict that can arise between prior assumptions on component lifetimes
and observed behaviour of components in the system under study.
As we will show in Section~\ref{sec:weibull}, when taking the standard choice of a conjugate prior,
prior-data conflict is ignored, as the spread of the posterior distribution does not increase in case of such a conflict,
ultimately conveying a false sense of certainty
by communicating that we can quantify the reliability of a system quite precisely when in fact we can not.
%
In contrast, our method will indicate prior-data conflict by wider bounds for $R_\text{sys}(t)$.
This behaviour is obtained by a specific choice for the set of priors (see \citeNP{Walter2009a} and \citeNP{diss} \S 3.1.4)
which leads to larger sets of posterior distributions when prior knowledge and data are in conflict
(see Section~\ref{sec:modforsurpr} for more details).

As a first step towards more realistic models, we apply this method
to a simplified parallel system with exchangeable components,
where component failures follow independent and identical Weibull distributions irrespective of their load.
We exemplarily consider the problem of predicting the reliability of a currently running one of a kind system,
where we have vague prior information on the lifetimes of the components the system is made of,
and the only available data consist of observed behaviour of the system components so far,
that is, the failure times of the components that have already failed,
and the fact that the remaining components still function,
whose failure time is thus right-censored.
We study the effect of surprisingly early or late component failures,
%on the system reliability prediction,
showing that observations in conflict to prior assumptions
indeed lead to more cautious system reliability predictions.

The paper is organized as follows.
In Section~\ref{sec:weibull}, we describe a Bayesian analysis for Weibull component lifetimes
and illustrate the issue of prior-data conflict.
Section~\ref{sec:modforsurpr} then details the use of sets of priors
for the scale parameter $\lambda$ of the Weibull distribution,
showing how this mitigates the prior-data conflict issue for $\lambda$.
In Section~\ref{sec:andrewsresults}, we present our results for the reliability estimation for a parallel system,
along with three examples that illustrate the desired behaviour.
Section~\ref{sec:conclusion} concludes the paper by summarizing the achievements and discussing avenues for further research.


\section{BAYESIAN ANALYSIS OF WEIBULL LIFETIMES}
\label{sec:weibull}

For each of the component lifetimes $T_i$, $i=1,\ldots,\ell$,
we assume a Weibull distribution with fixed shape parameter $k > 0$, with density and cdf%
\footnote{Our approach would be possible also for other parametric lifetime distributions
that form a canonical exponential family
(see, e.g., \citeNP[p.~202 and 272f]{2000:bernardosmith}, or \citeNP[p.~8]{diss}).}
\begin{align}
p_c(t \mid \lambda) &= \frac{k}{\lambda} t^{k-1} e^{-\frac{t^k}{\lambda}}\,, 
&
F_c(t \mid \lambda) &= 1 - e^{-\frac{t^k}{\lambda}} \,,
\label{eq:weibull-pdf-cdf}
\end{align}
where $\lambda > 0$ and $t > 0$.%
\footnote{The shape parameter $k$ determines whether the hazard rate is increasing ($k > 1$)
or decreasing ($k < 1$) over time.
For $k=1$, we obtain the Exponential distribution with constant hazard rate as a special case.
The value for $k$ will thus often be clear from practical considerations.}
The scale parameter $\lambda$ can be interpreted through the relation
\begin{align}
\E[T_i \mid \lambda] &= \lambda^{1/k}\, \Gamma(1 + 1/k)\,.
\label{eq:lambdainterpret}
\end{align}
For encoding expert knowledge about the reliability of the components,
we need to assign a prior distribution over the scale parameter $\lambda$.
A convenient choice is to use the inverse Gamma distribution,
commonly parametrized in terms of the hyperparameters $\alpha > 0$ and $\beta > 0$:
\begin{align}
p(\lambda \mid \alpha, \beta) &= \frac{\beta^\alpha}{\Gamma(\alpha)} \lambda^{-(\alpha + 1)} e^{-\beta/\lambda} \,,
\end{align}
in short, $\lambda \mid \alpha, \beta \sim \ig(\alpha, \beta)$.
The inverse Gamma is convenient because it is a conjugate prior,
i.e., the posterior obtained by Bayes' rule is again inverse Gamma and thus easily tractable;
the prior parameters only need to be updated to obtain the posterior parameters.

In the standard Bayesian approach, 
one has to fix a prior by choosing values for $\alpha$ and $\beta$
to encode specific prior information about component lifetimes.
In our imprecise approach, we allow instead these parameters
to vary in a set, this is advantageous also
because expert knowledge is often vague,
and it is difficult for the expert(s) to pin down precise hyperparameter values.
For the definition of the hyperparameter set,
we use however a parametrization in terms of $\nz > 1$ and $\yz > 0$ instead of $\alpha$ and $\beta$,
\begin{align}
\nz &= \alpha - 1\,,
&
\yz &= \beta / \nz \,,
\end{align}
where $\yz$ can be interpreted as the prior guess for the scale parameter $\lambda$,
as $\E[\lambda\mid\nz,\yz] = \yz$.
This parametrization also makes the nature of the combination
of prior information and data through Bayes' rule more clear:
After observing $n$ component lifetimes $\vec{t} = (t_1, \ldots, t_n)$,
the updated parameters are
\begin{align}
\nn &= \nz + n\,, 
&
\yn &=  \frac{\nz \yz + \taut}{\nz + n}\,,
\label{eq:ig-update}
\end{align}
where $\taut = \sum_{i=1}^n t_i^k$. %
%\footnote{
We thus have
\begin{align}
\lambda \mid \nz, \yz, \vec{t} \sim \ig(\nz + n + 1, \nz \yz + \taut). %}
\label{eq:ig-update-alpha}
\end{align}
From the simple update rule \eqref{eq:ig-update}, we see that
$\yn$ is a weighted average of the prior parameter $\yz$ and the Maximum Likelihood estimator $\taut/n$,
with weights $\nz$ and $n$, respectively.
$\nz$ can thus be interpreted as a prior strength or pseudocount,
indicating how much our prior guess should weigh against the $n$ observations.
Furthermore, $\V[\lambda\mid\nz,\yz] = (\yz)^2 / (1 - 1/\nz)$,
so for fixed $\yz$, the higher $\nz$,
the more probability mass is concentrated around $\yz$. %$p(\lambda\mid\nz,\yz)$ 

However, the weighted average structure for $\yn$
is behind the problematic behaviour in case of prior-data conflict.
Assume that from expert knowledge we expect
to have a mean component lifetime of 9 weeks.
Using \eqref{eq:lambdainterpret}, with $k=2$ we obtain $\yz = 103.13$.
We choose $\nz = 2$, so our prior guess for the mean component lifetime
counts like having two observations with this mean.
If we now have a sample of two observations
with surprisingly early failure times $t_1 = 1$ and $t_2 = 2$,
using \eqref{eq:ig-update} we get $\nfun{2} = 4$
and $\yfun{2} = \frac{1}{4}(2 \cdot 103.13 + 1^2 + 2^2) = 52.82$,
so our posterior expectation for the scale parameter $\lambda$ is $52.82$,
equivalent to a mean component lifetime of $6.44$ weeks.
The posterior standard deviation (sd) for $\lambda$ is $60.99$.
Compared to the prior standard deviation of $145.85$,
the posterior expresses now more confidence that mean lifetimes are around $\yfun{2} = 52.82$
than the prior had about $\yz = 103.13$.
This irritating conclusion is illustrated in Figure~\ref{fig:weibull-pdc};
the posterior cdf is shifted halfway towards the values for $\lambda$
that the two observations suggest
(the ML estimator for $\lambda$ would be $2.5$),
and is steeper than the prior (so the pdf is more pointed),
thus conveying a false sense of certainty about $\lambda$.%
\footnote{This is a general problem in Bayesian analysis with canonical conjugate priors.
For such priors, the same update formula \eqref{eq:ig-update} applies,
and so conflict is averaged out, for details see \citeN{Walter2009a} and \citeN[\S 3.1.4 and \S A.1.2]{diss}.}
We would obtain almost the same %values of $\nfun{2}$ and $\yfun{2}$, and so the same 
posterior distribution
if we had assumed the mean component lifetime to be 7 weeks (so $\yz = 62.39$),
and observed lifetimes $t_1 = 6$, $t_2 = 7$ in line with our expectations.
It seems unreasonable to make the same probability statements on component lifetimes in these two fundamentally different scenarios.

\begin{figure}
\includegraphics[width=0.48\textwidth]{fig1}
\caption{Prior and posterior cdf for $\lambda$ given surprising observations;
the conflict between prior assumptions and data is averaged out,
with a more pointed posterior giving a false sense of certainty.}
\label{fig:weibull-pdc}
\end{figure}


\section{MODELS REFLECTING SURPRISING DATA}
\label{sec:modforsurpr}

Despite this issue of ignoring prior-data conflict,
the tractability of the update step
(the posterior is again a parametric distribution)
is a very attractive feature of the conjugate setting.
As was shown by \citeN{Walter2009a} (see also \citeNP{diss}, \S 3.1),
it is possible to retain tractability and to have a meaningful reaction to prior-data conflict
when using sets of priors generated by varying both $\nz$ and $\yz$.
Then, the magnitude of the set of posteriors,
and with it the precision of posterior probability statements,
will be sensitive to the degree of prior-data conflict,
i.e.\ leading to more cautious probability statements when prior-data conflict occurs.

Instead of a single prior guess $\yz$ for the mean component lifetimes,
we will now assume a range of prior guesses $[\yzl, \yzu]$, and also a range $[\nzl, \nzu]$ of pseudocounts,
i.e., we now consider the set of priors
%\begin{align}
\begin{multline}
\MZ := \{ p(\lambda\mid\nz,\yz) \mid \nz \in [\nzl, \nzu],\\ \yz \in [\yzl, \yzu] \}
\label{eq:setofpriors}
\end{multline}
%\end{align}
to express our prior knowledge about component lifetimes.
Each of the priors $p(\lambda\mid\nz,\yz)$ is then updated to the posterior
$p(\lambda\mid\nz,\yz,\vec{t}) = p(\lambda\mid\nn,\yn)$
by using \eqref{eq:ig-update},
such that the set of posteriors $\MN$ can be written as
$\MN = \{ p(\lambda\mid\nn,\yn) \mid \nz \in [\nzl, \nzu], \yz \in [\yzl, \yzu] \}$.
This procedure of using Bayes' Rule element by element
is seen as self-evident in the robust Bayesian literature,
but can be formally justified as being \emph{coherent}
(a self-consistency property)
in the framework of imprecise probability, where it is known as
\emph{Generalized Bayes' Rule} \cite[\S 6.4]{1991:walley}.

Technically, it is crucial to consider a range of pseudocounts $[\nzl, \nzu]$
along with the range of prior guesses $[\yzl, \yzu]$,
as only then $\taut/n \not\in [\yzl, \yzu]$
leads to the set of posteriors being larger
and hence reflecting prior-data conflict.

Continuing the example from Section~\ref{sec:weibull} and Figure~\ref{fig:weibull-pdc},
assume now for the mean component lifetimes the range 9 to 11 weeks,
this corresponds to $[\yzl,\yzu] = [103.13, 154.06]$.
We also choose $[\nzl,\nzu] =[2, 5]$,
so we value or prior information as equivalent to having seen two to five observations.
Compare now the set of posteriors obtained from observing
$t_1 = 1$, $t_2 = 2$ (as before), see Figure~\ref{fig:setofpost-pdc},
and $t_1 = 10$, $t_2 = 11$, see Figure~\ref{fig:setofpost-nopdc}.
We now have a clear difference between the two scenarios of
observations in line with expectations (Figure~\ref{fig:setofpost-nopdc})
and observations in conflict (Figure~\ref{fig:setofpost-pdc}).
In Figure~\ref{fig:setofpost-pdc}, the set of posteriors (blue)
is shifted towards the left, but has about the same size as the set of priors (red),
and so posterior quantification of reliability has the same precision,
despite having seen two observations.
Instead, in Figure~\ref{fig:setofpost-nopdc}, the set of posteriors
is smaller than the set of priors,
such that the two observations have increased the precision of probability statements.

\begin{figure}
\includegraphics[width=0.48\textwidth]{fig2}
\caption{Set of prior (red) and posterior cdfs (blue) for $\lambda$ for two surprising observations $t_1 = 1$, $t_2 = 2$.}
\label{fig:setofpost-pdc}
\end{figure}

\begin{figure}
\includegraphics[width=0.48\textwidth]{fig3}
\caption{Set of prior (red) and posterior cdfs (blue) for $\lambda$ for two unsurprising observations $t_1 = 10$, $t_2 = 11$.}
\label{fig:setofpost-nopdc}
\end{figure}

%\comments{More details for Fig 2 and 3 example? Also prior set?}
%
As each posterior in $\MN$ corresponds to a predictive distribution for $T_\text{sys}$,
we will have a set of reliability functions $R_\text{sys}(t)$.
The derivation of $R_\text{sys}(t)$ for a parallel system with $\ell$ components
will be given in Section~\ref{sec:andrewsresults} below.
This will include, in contrast to previous studies using sets of priors of type 
\eqref{eq:setofpriors}, the treatment of censored observations.%
\footnote{More specifically, we consider the case of \emph{non-informative right censoring},
where the censoring process is independent of the failure process.}


\section{ROBUST BAYESIAN INFERENCE FOR A PARALLEL SYSTEM}
\label{sec:andrewsresults}

We consider a simplified parallel system with $\ell$ exchangeable components,
so the system functions when at least one of the $\ell$ components functions.
Exchangeability means that the components are of the same type
and that any permutation of the component labels would not change any results.
We therefore assume that component load does not influence the failure distribution.%
\footnote{The system can thus be seen as either a standby parallel with perfect switch
where components degrade irrespective of load,
or as an active parallel system where again load is not relevant for the failure distribution.
We plan to extend our model to cover the influence of load, see our comments in Section~\ref{sec:conclusion}.}
We observe the system until a certain time point $t_\text{now}$
and wish to calculate $R_\text{sys}(t)$ for $t > t_\text{now}$.

From the components in the system that have already failed by the current time $t_\text{now}$,
we can learn about the failure distribution of the remaining components.
But also the fact that these remaining components have not failed yet
contains information on the lifetime distribution $p_c(t\mid \lambda)$.
Such observations for which the failure event has not happened yet are called \emph{right-censored}.
Right-censoring is very common in reliability studies,
%and data including right-censored observations are often encoded
%in pairs $(t_i, \delta_i)$, where $\delta_i$ indicates whether
%a failure was observed at time $t_i$ for item $i$ ($\delta_i = 1$)
%or the item $i$ was lost for follow-up at time $t_i$ ($\delta_i = 0$).
%Often, censored observations are alternatively marked as $t_i^+$, in contrast to observed failures $t_i$.
and censored observations are usually marked as $t_i^+$, in contrast to observed failures $t_i$.
%The posterior for a parameter $\lambda$ based on $n$ possibly censored observations
%$(t_i, \delta_i)$, $i=1,\ldots,n$,
%can be written as (see, e.g., \citeNP{2003:lawless}, \S 2.2)
%\begin{align}
%p(\lambda \mid \{(t_i, \delta_i)\}^n) \propto p(\lambda) \prod_{i=1}^n p(t_i)^{\delta_i} (1-F(t_i))^{1-\delta_i}\,.
%\end{align}
The posterior distribution for parameter $\lambda$
based on $\ell$ observations $\vec{t}_m^\ell := (t_1,\ldots,t_m,t^+_\text{now},\ldots,t^+_\text{now})$
(so we have $m$ observed failure times $t_1,\ldots,t_m$
and $\ell-m$ right-censored observations $t^+_\text{now}$
for the $\ell-m$ components that have not failed as of current time $t_\text{now}$)
can be written as%
\footnote{See, e.g., \citeN[\S 2.2]{2003:lawless} for the derivation of the likelihood in case of right-censored observations.}
\begin{align}
p(\lambda \mid \vec{t}_m^\ell) \propto p(\lambda) \prod_{j=1}^m p_c(t_j) \big(1-F_c(t_\text{now})\big)^{\ell-m}\,,
\end{align}
where $p(\lambda)$ is some prior distribution over $\lambda$.%
\footnote{Of course, one can take as prior $p(\lambda)$ also a posterior distribution
obtained from updating a prior with data from, e.g., component tests,
but we leave this possibility out here to keep the notation simple.}
%
%Also with censored observations, choosing $\lambda \sim \ig(\alpha,\beta)$
%leads to the posterior $p(\lambda \mid \alpha,\beta, \vec{t}_m^n) = p(\lambda \mid \nz, \yz, \vec{t}_m^n)$
%being inverse Gamma again:
Choosing $\lambda \sim \ig(\alpha,\beta)$ and writing the posterior in terms of $\nz$ and $\yz$, we get
\begin{align}
\lefteqn{p(\lambda \mid \nz, \yz, \vec{t}_m^\ell)} \\
	&\propto \prod_{j=1}^{m} p_c(t_j \mid \lambda) \big(1 - F_c(t_\text{now})\big)^{\ell-m} p(\lambda\mid \nz, \yz) \\
%	&\propto \frac{k^m}{\lambda^m} \bigg(\prod_{j=1}^{m} t_j^k \bigg)
%             e^{-\frac{1}{\lambda}\sum_{j=1}^{m} t_j^k}
%             e^{-\frac{1}{\lambda}(\ell-m) t_\text{now}^k} p_c(\lambda\mid \nz, \yz) \\
	&\propto \lambda^{-m} e^{-\frac{1}{\lambda}(\sum_{j=1}^{m} t_j^k + (\ell-m) t_\text{now}^k)}
           \lambda^{-(\nz+1)-1}   e^{-\frac{1}{\lambda}(\nz \yz)} \\
	&\propto \lambda^{-(\nz+m+1)-1} e^{-\frac{1}{\lambda}((\nz \yz + (\ell-m) t_\text{now}^k + \sum_{j=1}^{m} t_j^k))} \,.
\end{align}
This forms the core of an inverse gamma distribution, and thus we have
$\lambda \mid \nz, \yz, \vec{t}_m^\ell \sim \ig\left(\nz + m + 1, \nz \yz + (\ell-m) t_\text{now}^k + \sum_{j=1}^{m}t_j^k \right)$.
Also for data including right-censored observations,
the posterior is again inverse gamma and posterior inference remains tractable.
Comparing with \eqref{eq:ig-update-alpha},
we see that $\alpha$ is now incremented only by the observed failure events,
while $\beta$ is updated using also the censoring times $t_\text{now}$.

For $R_\text{sys}(t \mid \vec{t}_m^\ell)$, first note that we know that the system has functioned until $t_\text{now}$,
so $R_\text{sys}(t \mid \vec{t}_m^\ell) = 1$ $\forall t \leq t_\text{now}$.
For $t > t_\text{now}$, let us consider the probability
that the system fails before $t$ given data $\vec{t}_m^\ell$ and $\lambda$,
\begin{align}
\lefteqn{P(T_\text{sys} \leq t \mid \vec{t}_m^\ell, \lambda)} \\ 
 &= P(T_{m+1}\leq t \cap \ldots \cap T_\ell \leq t \mid \vec{t}_m^\ell, \lambda) \\
 &= \big( P(T_{m+1} \leq t \mid \vec{t}_m^\ell, \lambda) \big)^{\ell-m} \\
 &= \big( P(T_{m+1} \leq t \mid T_{m+1} > t_\text{now}, \lambda) \big)^{\ell-m} \\
 &= \big( 1 - e^{-\frac{1}{\lambda} (t^k - t_\text{now}^k)} \big)^{\ell-m}\,,
\end{align}
where we used the definition of conditional probability
and Eq.~\eqref{eq:weibull-pdf-cdf} in the last step.
%
Then, the predictive probability
$P(T_\text{sys} \leq t \mid \vec{t}_m^\ell) = 1 - R_\text{sys}(t \mid \vec{t}_m^\ell)$ is
\begin{align}
\lefteqn{P(T_\text{sys} \leq t \mid \vec{t}_m^\ell)} \\
% &= \int_0^\infty P(T_{m+1}\leq t \cap \ldots \cap T_\ell \leq t \mid \vec{t}_m^\ell, \lambda)
%    p(\lambda \mid \nz, \yz, \vec{t}_m^\ell) \dd\lambda \\
% &= \int_0^\infty \big( P(T_{m+1} \leq t \mid \vec{t}_m^\ell, \lambda) \big)^{\ell-m}
%    p(\lambda \mid \nz, \yz, \vec{t}_m^\ell) \dd\lambda \\
% &= \int_0^\infty \big( P(T_{m+1} \leq t \mid T_{m+1} \leq t_\text{now}, \lambda) \big)^{\ell-m}
%    p(\lambda \mid \nz, \yz, \vec{t}_m^\ell) \dd\lambda \\
 &= \int_0^\infty P(T_\text{sys} \leq t \mid \vec{t}_m^\ell, \lambda)
    p(\lambda \mid \nz, \yz, \vec{t}_m^\ell) \dd\lambda \\
 &= \int_0^\infty \big( 1 - e^{-\frac{1}{\lambda} (t^k - t_\text{now}^k)} \big)^{\ell-m}
    p(\lambda \mid \nz, \yz, \vec{t}_m^\ell) \dd\lambda \\
 &=\sum_{i=0}^{\ell-m} (-1)^i {\ell-m \choose i} \times \\ 
 & \qquad
    \int_0^\infty e^{-\frac{i}{\lambda} (t^k - t_\text{now}^k)}
    p(\lambda \mid \nz, \yz, \vec{t}_m^\ell) \dd\lambda \,.
\end{align}
Substituting in the posterior, we obtain for the integral
\begin{align}
\lefteqn{%
    \int_0^\infty e^{-\frac{i}{\lambda} (t^k - t_\text{now}^k)} %
    p(\lambda \mid \nz, \yz, \vec{t}_m^\ell) \dd\lambda } \\
 &= \frac{\left(\nz \yz + (\ell-m) t_\text{now}^k + \sum_{j=1}^{m} t_j^k \right)^{\nz + m + 1}}{\Gamma(\nz + m + 1)}\times \\
 &  \int_0^\infty\!\! e^{-\frac{i(t^k - t_\text{now}^k)}{\lambda}} \lambda^{\nz + m + 1}
    e^{-\frac{\nz \yz + (\ell-m) t_\text{now}^k + \sum_{j=1}^m t_j^k}{\lambda}}\! \dd\lambda\,.
\end{align}
The terms remaining below the integral form the core of an inverse Gamma distribution
with parameters
\begin{align}
\alpha &= \nz + m + 1\,,\\
\beta  &= \nz \yz + (\ell - m - i) t_\text{now}^k + \sum_{j=1}^{m} t_j^k + i t^k\,,
\end{align}
allowing us to solve the integral using the corresponding normalization constant.
%
Substituting all back in, we thus obtain
\begin{align}
\label{eq:generalformula}
\lefteqn{%
R_\text{sys}(t \mid \vec{t}_m^\ell)
  = 1 - \sum_{i=0}^{\ell-m} (-1)^i {\ell-m \choose i} \, \times } & \\
 & %\qquad
   \left(\!\! \frac{\nz \yz + (\ell-m)   t_\text{now}^k + \sum_{j=1}^{m} t_j^k}%
                   {\nz \yz + (\ell-m-i) t_\text{now}^k + \sum_{j=1}^{m} t_j^k + i t^k} \!\!\right)^{\!\!\!\nz + m + 1}.\nonumber
\end{align}
%We can see that for $m$ increasing, i.e., the more failures we observe among the $\ell$ components, we lose terms in our sum.
%We also see that each failure adds an extra power to the fraction.
%The censored observations come into play in the fraction, and their effect decreases in each term as $i$ increases. 
%Besides the $\nz$ in the exponent, the prior parameters $\nz$ and $\yz$ appear only in form of their product $\nz\yz$.
%
As described in Section~\ref{sec:modforsurpr},
we can now choose bounds $[\nzl, \nzu]$ and $[\yzl, \yzu]$
to reflect our prior knowledge about component lifetimes.
The bounds $\lRsys(t \mid \vec{t}_m^\ell)$ and $\uRsys(t \mid \vec{t}_m^\ell)$ for the reliability function,
which are obtained by minimizing and maximizing \eqref{eq:generalformula}
over $[\nzl, \nzu] \times [\yzl, \yzu]$ for each $t > t_\text{now}$,
will adequately reflect our knowledge about the reliability quantification of the system,
based on both the prior knowledge and the behaviour of the components in the system until $t_\text{now}$.
Specifically, we illustrate the quantification in case of prior-data conflict through three examples,
where we consider a parallel system with three components
for which we may observe surprising component behaviour.

For all three examples, we continue to use the prior assumptions from Section~\ref{sec:modforsurpr},
(where we chose $[\nzl, \nzu] = [2,5]$ and $[\yzl, \yzu] = [103.13, 154.06]$,
expecting the mean component lifetimes to be in the range of 9 to 11 weeks),
so the set of prior cdfs $\Fsys(t)$ is identical in the three Figures~\ref{fig:exampletwofailuresinline} to \ref{fig:examplenofailures} below.
For easy comparison between the examples, all three plots depict the range $0 \leq t \leq 70$.

\begin{example}
\label{ex:1}
% two failures in line with expectation. Post2Finline.pdf 
Assume that we observe the system until $t_\text{now} = 11$
and that we have seen failures for two of the three components
at $t_1=10$ and $t_2=11$, which are thus in line with our expectations.
We therefore have $\vec{t}_2^3 = (10, 11, 11^+)$.
(Our set of posteriors for the Weibull scale parameter $\lambda$ would thus be like in Figure~\ref{fig:setofpost-nopdc}
were it not for the additional censored observation $11^+$.)
The prior bounds $\lFsys(t)$ and $\uFsys(t)$ (red),
and the resulting posterior bounds $\lFsys(t \mid \vec{t}_2^3)$ and $\uFsys(t \mid \vec{t}_2^3)$ (blue),  
are depicted in Figure~\ref{fig:exampletwofailuresinline}.
Notice that for all posterior cdfs between the bounds
it holds that $\Fsys(t \mid \vec{t}_2^3) > 0$ only for $t > t_\text{now} = 11$,
as the system has not failed by $t_\text{now} = 11$.
We can see that based only on prior assumptions (so at time $0$),
we predict, e.g., $\Fsys(15) = P(T_\text{sys} \leq 15) = [0.522, 0.752]$.
After observing $\vec{t}_2^3$, i.e.\ at current time $11$,
we have instead $\Fsys(15 \mid \vec{t}_2^3) = P(T_\text{sys} \leq 15 \mid \vec{t}_2^3) = [0.511, 0.600]$.
As component failures were in line with our expectations,
bounds for $\Fsys(15)$, and in fact for the whole range of $t$, 
have become narrower,
such that we can quantify the system reliability now more precisely.
\end{example}

\begin{figure}
\includegraphics[width=0.48\textwidth]{Post2Finline}
\caption{Set of prior (red) and posterior (blue) cdfs $F_\text{sys}(t)$ for two failures $t_1 = 10$, $t_2 = 11$ in line with expectations.}
\label{fig:exampletwofailuresinline}
\end{figure}

\begin{example}
\label{ex:2}
%***two early failures. Post2FPlot.pdf
Assume that we observe now instead two surprisingly early failures at times $t_1 = 1$ and $t_2 = 2$
(just like we assumed for the posterior on $\lambda$ in Figure~\ref{fig:setofpost-pdc}).
The posterior bounds $\lFsys(t \mid \vec{t}_2^3)$ and $\uFsys(t \mid \vec{t}_2^3)$
for $\vec{t}_2^3 = (1, 2, 2^+)$ 
are depicted in Figure~\ref{fig:exampletwoearlyfailures} (blue).
Now, the prediction for the system to fail before 15 weeks
is $[\lFsys(15 \mid \vec{t}_2^3), \uFsys(15 \mid \vec{t}_2^3)] = [0.864, 0.971]$.
This interval contains not only higher values,
but its length has not reduced as much as was the case in Example~\ref{ex:1}.
Also, the posterior predicted survival probability until $t=10$
is $[\lRsys(10 \mid \vec{t}_2^3), \uRsys(10 \mid \vec{t}_2^3)] = [0.158,0.395]$,
similarly cautious as the prior interval $[0.591,0.830]$
despite two observed failure events. 
\end{example}

\begin{figure}
\includegraphics[width=0.48\textwidth]{Post2FPlotpaper}
\caption{Set of prior (red) and posterior (blue) cdfs $F_\text{sys}(t)$ for two surprisingly early failures $t_1 = 1$, $t_2 = 2$.}
\label{fig:exampletwoearlyfailures}
\end{figure}

\begin{example}
\label{ex:3}
%***no failures by week 15. PostPredGS.pdf
Assume now that we do not observe any failures until week $15$,
such that we have $\vec{t}_0^3 = (15^+, 15^+, 15^+)$.
Prior and posterior bounds for $\Fsys$ are depicted in Figure~\ref{fig:examplenofailures}.
The posterior set of pdfs is shifted to the right, but also larger than the prior set,
so clearly indicating prior-data conflict through wide probability bounds.
E.g., before the start of the system, the predicted probability for the system to have failed by time $20$
was $\Fsys(20) = [0.794, 0.914]$, whereas now it is $\Fsys(20 \mid \vec{t}_0^3) = [0.091,0.210]$.
Had we scheduled maintenance work for week 20
based on the probability for system failure being between 55\% and 75\%,
we could now reschedule it to, e.g, week 29,
for which we predict $\Fsys(29 \mid \vec{t}_0^3) = [0.523,0.793]$.
\end{example}

\begin{figure}
\includegraphics[width=0.48\textwidth]{PostPredGSpaper}
\caption{Set of prior (red) and posterior (blue) cdfs $F_\text{sys}(t)$ given no failures observed by week 15.}
\label{fig:examplenofailures}
\end{figure}

\section{CONCLUSION AND OUTLOOK}
\label{sec:conclusion}

In this paper we presented a robust Bayesian approach to reliability estimation for a simple parallel system,
and showed how the use of sets of prior distributions results in increased imprecision,
i.e., more cautious probability statements, in case of prior-data conflict (see Examples~\ref{ex:2} and \ref{ex:3}),
while giving more precise reliability bounds when prior and data are in agreement (see Example~\ref{ex:1}).
The parameters through which prior information is encoded have a clear interpretation
and are thus easily elicited.
Calculation of lower and upper predictive reliability bounds is tractable,
requiring only a simple two-dimensional box-constrained optimization of Equation~\eqref{eq:generalformula}.

%The robust Bayesian setting we employed provides many further modelling opportunities
%for which we think increased imprecision is an appropriate tool.
We think that increased imprecision is an appropriate tool for mirroring prior-data conflict
when considering sets of priors as is done in both the robust Bayesian and imprecise probability framework.
We want to emphazise, however, that this tool may be useful already
for just highlighting `conflict' between multiple information sources,
and that we do not think that the resulting set of posteriors,
although it can form a meaningful basis, must necessarily be used
for all consequential inferences, as a strict Bayesian would posit.
We believe an analyst is free to reconsider any aspect of a model
(of which the choice of prior can be seen to form a part) after seeing the data,
and so may use our method only for becoming aware of a conflict between prior and data.

The robust Bayesian setting we employed provides many further modelling opportunities
beyond the explicit reaction to prior-data conflict.
These opportunities have not yet been explored and
provide a wide field for further research.
An example is our currently ongoing investigation into
extending the present model to allow also for an appropriate reflection of very strong agreement between prior and data.

There are many aspects to further develop in our analysis and modeling.
Straightforward to implement is, e.g., the analysis of the effect of replacing failed components
in the parallel system on the reliability quantification $[\lRsys(t \mid \vec{t}), \uRsys(t \mid \vec{t})]$.
Criteria for the trade-off between the cost of replacement and the gain in reliability
would have to be adapted for the interval output of our model,
leading to very interesting research questions.

Following the approach in \citeN[\S 5]{2014:bayessurvsign},
we intend to generalize our model to arbitrary system structures
by making use of the recently introduced survival signature \cite{2012:survsign}.
Like the system signature, the survival signature makes it possible to separate the system structure
from individual component reliability, such that temporal aspects need to be accounted for
only through the component lifetime distributions.
Furthermore, it allows, in contrast to the system signature,
for an easy calculation of the reliability for systems with multiple types of components.
Having multiple types of components will allow us, among other improvements,
to model component loads more realistically.
%
We also plan to explore the nonparametric lifetime models
described in \citeN[\S 4]{2014:bayessurvsign},
which allow to relax the rather strong assumptions in our parametric modeling of component lifetimes,
but require either more data, or a much more detailed assessment of prior knowledge, than our approach.
%nonparametric lifetime as in \citeNP{2014:bayessurvsign} \S 4, \comments{only mention if space}

Another important aspect in system reliability we have not accounted for yet
is the possibility of common-cause failures,
i.e., failure events where several components fail at the same time due to a shared or common root cause.
We thus plan to combine the common-cause failure model approaches
of \citeN{Troffaes2014a} and \citeN{2015:coolen-commoncause}.

On a more abstract level, our choice for the set of prior parameters
(generating the set of prior component failure distributions) as $[\nzl, \nzu] \times [\yzl, \yzu]$
has the advantage of allowing for easy elicitation and tractable inferences,
but it may not be suitable to reflect certain kinds of prior knowledge.
Also, as studied in \cite{Walter2011a} and \cite[\S 3.1]{diss},
the shape of the prior parameter set has a crucial influence on model behaviour
like the severity of prior-data conflict reaction.
As noted in \cite[pp.~66f]{diss},
more general prior parameter set shapes are possible in principle,
but may be more difficult to elicit and make calculations more complex.


\bibliographystyle{chicaco}
\bibliography{esrel2015refs}

\end{document}

